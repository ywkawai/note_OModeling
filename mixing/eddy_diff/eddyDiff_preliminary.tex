\subsection{準備}
今, 
%%
\begin{equation}
  \DD{\phi}{t} = \nabla \cdot (\kappa_m \nabla \phi)
\end{equation}
%%
なる方程式に従う, トレサー$\phi$を考える. 
ここで, $\kappa_m$は分子拡散係数である. 
もし移流速度が非発散であるならば, アンサンブル平均した方程式は, 
分子拡散を無視するとき, 
%%
\begin{equation}
  \DD{\overline{\phi}}{t} = - \nabla \cdot \overline{\Dvect{v}' \phi'}
\end{equation}
%%
となる. 
渦輸送を拡散によってパラメータ化するならば, そのとき, 
%%
\begin{equation}
  \Dvect{F} = \overline{\Dvect{v}' \phi'}
            = - \Dvect{K} \nabla \phi
\end{equation}
%%
と書ける. 
ここで, $\Dvect{K}$は一般に二階テンソルである. 
もし渦による項を拡散によってパラメータ化するならば, 
生じる問題は次の二点である. 
\begin{itemize}
 \item 渦拡散係数の\textbf{大きさ}. 可能性としては, 平均流の関数である. 
 \item 拡散係数テンソルの\textbf{構造}. 特に, 拡散テンソルの対称部分と反対称部分の分離構造. 
\end{itemize}
%%


