%%
分子拡散の効果を除いて自由に時間発展するトレサーは, 
%%
\begin{equation}
  \DD{\phi}{t} = \nabla \cdot (\kappa_m \nabla \phi))
\label{eq:tracer_eq_molecular_diff}
\end{equation}
%%
に従う. 
ここで, $\kappa_m$は分子拡散係数であり, 正のスカラー量である. 
より一般的には, 
%%
\begin{equation}
  \DD{\phi}{t} = - \nabla \cdot \Dvect{F} = \nabla \cdot \Dvect{K} \nabla \phi
\label{eq:tracer_eq_general_diff}
\end{equation}
%%
と書かれる. 
ここで, $\Dvect{K}$は($\phi$がスカラーであれば)2階テンソルであり, 
$\Dvect{F}=-\Dvect{K}\nabla \phi$は$\phi$の拡散フラックスである. 
そのフラックスは, $\phi$の等値面を横切る成分(拡散フラックスと呼ばれる)と, 
沿う成分(スキューフラックスと呼ばれる)をもつ. 
これらのフラックスはそれぞれ, 拡散係数フラックスの対称成分$\Dvect{K}$と反対称成分$\Dvect{A}$に関係している. 
ここで, 
%%
\begin{equation}
  \Dvect{K} = \Dvect{S} + \Dvect{A}
\end{equation}
%%
であり, 成分表示を用いて, 
%%
\begin{equation}
 S_{mn} = \dfrac{1}{2} (K_{mn} + K_{nm}), \;\;\;
 A_{mn} = \dfrac{1}{2} (K_{mn} - K_{nm}) 
\end{equation}
%% 
である. 
反対称テンソルの対角要素は, ゼロである. 
これらの二種類のテンソルによってもたらされる輸送は, 異なる物理的特徴をもつことを以下で述べる. 

\subsection{対称テンソルによる輸送(拡散)}

\subsubsection*{最も簡単な例}
最も簡単な場合は媒質が等方的な場合であり, このとき$\Dvect{K}$は対角的かつ要素の値は同じである. 
すなわち, 
%%
\begin{equation}
  \Dvect{K} = \Dvect{S}
\begin{pmatrix}
 \kappa &0 &0 \\
 0 &\kappa &0 \\
 0 &0 &\kappa
\end{pmatrix}
\end{equation}
%%
と書かれ, よく知られた$\Dvect{F}=-\kappa\nabla \phi$を得る. 
したがって, \eqref{eq:tracer_eq_general_diff}は\eqref{eq:tracer_eq_molecular_diff}と同じ形式をもつ. 
もし$\kappa$が正ならば, そのときフラックスはたとえ$\kappa$が空間非一様であっても, 
\textbf{勾配下向き}(downgradient)である. すなわち, 
%%
\begin{equation}
  \Dvect{F} \cdot \nabla \phi \le 0. 
\end{equation}
%%
さらに, そのような拡散は分散が散逸する. 
それを確かめるために, 運動方程式
%%
\begin{equation}
 \DP{\phi}{t} = \nabla \cdot (\kappa \nabla \phi)
\end{equation}
%%
を考えよう. 
両辺に$\phi$を掛けて領域$V$に渡って積分し, 部分積分を実行すると, 
%%
\begin{equation}
   \dfrac{1}{2} \DD{}{t} \int_V \phi^2 \Dd{V}
 = \int_V \int_{V} \Dvect{F} \cdot \nabla \phi \Dd{V}
 = - \kappa (\nabla \phi)^2 \le 0
\end{equation}
%%
を得る. 
ここで, $\phi$物質の境界を横切るフラックスは存在しないことを仮定した. 
しかし, 拡散は場の一次モーメントを保存する. 
すなわち, 
%%
\begin{equation}
  \DD{}{t} \int \phi \Dd{V} = \int_V \nabla \cdot (\kappa \nabla \phi) \Dd{V} = 0 
\end{equation}
%%
である. 
ここで, 再び境界を横切るフラックスはゼロであることを仮定した. 

\subsubsection*{一般化}
対称な拡散係数テンソルがもたらす輸送は, 拡散フラックス
%%
\begin{equation}
  \Dvect{F}_d = - \Dvect{S} \nabla \phi
  = -S_{mn} \partial_n \phi
\end{equation}
%%
である. 
ここで, 添え字に対してアインシュタイン縮約規則を適用するものとする. 
一般的に, フラックス$\Dvect{F}_d$はトレサーの勾配と平行な成分をもつ. 
簡単化のために, 
%%
\begin{equation}
 \DP{\phi}{t} = - \nabla \cdot \Dvect{F}_d 
     = \nabla \cdot (\Dvect{S} \nabla \phi)
\end{equation}
%%
なる運動方程式を考えよう. 
この方程式は, 境界を横切るフラックスが存在しなければ, $\phi$の一次モーメントを保存する. 
トレサーの分散は, 
%%
\begin{equation}
   \dfrac{1}{2} \DD{}{t} \int_V \phi^2 \Dd{V}
 = \int_{V} \phi \nabla \cdot ( \Dvect{S} \nabla \phi) \Dd{V}
 = - \int_V (\Dvect{S}\nabla \phi) \cdot \nabla \phi \Dd{V}
\end{equation}
%%
によって時間発展する. 
これは, $\Dvect{S}$が半正定値, すなわち
%%
\begin{equation}
  \nabla \phi \Dvect{S} \nabla \phi = \partial_m \phi S_{mn} \partial_n \phi \ge 0
\end{equation}
%%
であるならば, 右辺は負もしくはゼロであることが示される. 
またこのとき, そのような拡散がもたらすフラックスは, 
%%
\begin{equation}
  \Dvect{F}_d \cdot \nabla \phi = -\Dvect{S}\nabla\phi \cdot \nabla \phi \le 0
\end{equation}
という意味で勾配下向きである. 

\subsection{スキューフラックスによる輸送}
反対称テンソルと関係した輸送は, $\phi$の勾配と直交する. 
すなわち, 勾配上向きでも勾配下向きでもない. 
そのフラックスは, 
%%
\begin{equation}
  \Dvect{F}_{sk} = - \Dvect{A} \nabla \phi
  = - A_{mn} \partial_n \phi
\label{eq:skew_flux1}
\end{equation}
%%
であり, 
よって, 
%%
\begin{equation}
  \Dvect{F}_{sk} \cdot \nabla \phi
  = - \Dvect{A} \nabla \phi \cdot \nabla \phi = - A_{mn} \partial_n \phi \partial_m \phi = 0
\end{equation}
%%
である. 
ここで, 最後の等号において, $\Dvect{A}$の反対称性(対称テンソルと反対称テンソルの縮約はゼロ)を用いた%
\footnote{
$A_{mn} (\partial_n \phi \partial_m)$において, カッコ内が対称テンソルであることに注意して$S_{mn}$とおけば, 
$$
A_{mn} S_{mn} = - A_{nm} S_{nm} = - A_{mn} S_{mn}
$$
となり, $A_{mn} S_{mn}=0$を得る. 
}.
この理由により, 反対称テンソルと関係した輸送は, スキュー\textbf{フラックス}やスキュー\textbf{拡散}と呼ばれる. 
前者はトレサーの勾配に垂直なフラックスに一般的に用いられる用語であり, 
後者はスキューフラックスが反対称的な拡散係数を用いてパラメータ化されるときに用いられる用語である. 
このことから, 今トレサーが
%%
\begin{equation}
  \DP{\phi}{t} = \nabla \cdot (\Dvect{A} \nabla \phi)
\end{equation}
%%
に従うならば, 両辺に$\phi$を掛け部分積分を実行すれば, トレサーの分散が保存することが示される. 
ただし, 境界でフラックスがゼロであることを仮定した. 
つまり, \textbf{スキュー拡散は変数の分散に影響を及ぼさない}. 
非発散流による移流は, この性質を同じようにもつ物理的過程である. 
スキュー拡散はそのような移流と物理的に等価であり, そ
こではスキュー拡散フラックスの発散は, 適切に選択された移流フラックスの発散と同じである. 
そのことを以下で述べる. 

今, 
%%
\begin{equation}
  \Dvect{F}_{ad} \equiv \tilde{\Dvect{v}} \phi
\end{equation}
%%
なる形式のフラックスを, トレサー$\phi$の移流フラックスとして定義する. 
ここで, $\tilde{\Dvect{v}}$は非発散なベクトル場である. 
このフラックスの発散は, 単に, 
%%
\begin{equation}
  \nabla \cdot \Dvect{F}_{ad} = \tilde{\Dvect{v}} \cdot \nabla \phi
\end{equation}
%%
である. 
場$\tilde{\Dvect{v}}$は\textbf{擬速度}あるいは\textbf{準速度}と呼ばれるだろう. 
それは速度のように作用するが, どの流体粒子の速度である必要は無い. 
$\tilde{\Dvect{v}}$は非発散であるので, 
%%
\begin{equation}
 \tilde{\Dvect{v}} = \nabla \times \Dvect{\psi}
\end{equation}
%%
となるような, ベクトルの流線関数$\Dvect{\psi}$を定義することができる. 
場$\Dvect{\psi}$は一意ではない. 
任意関数の勾配を$\Dvect{\psi}$に付け加えてもよい. 
つまり, もし$\Dvect{\psi}' = \Dvect{\psi} + \nabla \gamma$であるならば, 
このとき, $\tilde{\Dvect{v}}=\nabla \times \Dvect{\psi} = \nabla \times \Dvect{\psi}'$となるからである.
スカラー場$\gamma$は\textbf{ゲージ}と呼ばれ, その選択の自由度はゲージの自由度である. 

移流フラックス$\Dvect{F}_{ad}$は, スキューフラックス$\Dvect{F}_{sk}$と以下の式によって関係付けられる. 
%%
\begin{equation}
  \phi \tilde{\Dvect{v}} 
 = \phi \nabla \times \Dvect{\psi} 
 = \nabla \times (\phi \Dvect{\psi}) - \nabla \phi \times \Dvect{\psi}, 
\end{equation}
%%
もしくは, 
%%
\begin{equation}
  \Dvect{F}_{ad} = \Dvect{F}_{r} +\Dvect{F}_{sk}. 
\end{equation}
%%
ここで, $\Dvect{F}_r=\nabla \times (\phi \Dvect{\psi})$は, 発散を伴わない回転のフラックスである.  
また, 
%%
\begin{equation}
  \Dvect{F}_{sk} = - \nabla \phi \times \Dvect{\psi}
\label{eq:skew_flux2}
\end{equation}
%%
は, スキューフラックスである. 
$\nabla \cdot \Dvect{F}_r=0$であるので, 
%%
\begin{equation}
  \nabla \cdot \Dvect{F}_{ad} = \nabla \cdot \Dvect{F}_{sk}
\end{equation}
%%
である. 
スキューフラックス$-\nabla \phi \times \Dvect{\psi}$と移流フラックス$\phi \nabla \times \Dvect{\psi}$は, 
一般的には異なる大きさや向きを持つことに注意する必要がある. 
等しいのはそれらの発散のみである. 
もし\eqref{eq:skew_flux1},\eqref{eq:skew_flux2}で与えられるスキューフラックスの発散が同じであるならば, 
そのとき$\Dvect{\psi}$は反対称テンソル$\Dvect{A}$と関係付けられなければならない. 
\eqref{eq:skew_flux1}を用いて, 
%%
\begin{equation}
\begin{split}
   \nabla \cdot \Dvect{F}_{sk} 
   &= - \partial_m ( A_{mn} \partial_n \phi) \\
   &= - (\partial_n \phi) (\partial_m A_{mn}) - [A_{mn} \partial_n \partial_m \phi] \\
   &= - \partial_n (\phi \partial_m A_{mn}) + [\phi \partial_n \partial_m A_{mn}]
\end{split}
\end{equation}
%%
を得る. 
ここで, 角括弧内の量は$\Dvect{A}$の反対称性によりゼロである. 
しかし, スキューフラックスの発散は移流フラックスの発散と等しい. 
すなわち, 
%%
\begin{equation}
  \nabla \cdot \Dvect{F}_{sk} 
 = \nabla \cdot \Dvect{F}_{ad} 
 = \partial_n (\phi \tilde{v}_n)
\end{equation}
%%
であるので, $\Dvect{A}$と関係するスキュー速度は, 
%%
\begin{equation}
  \tilde{v}_n = - \partial_m A_{mn}
\label{eq:pseudovel_AntiSymTensor_relation}
\end{equation}
%%
によって与えられる. 
この速度は, $\partial_n \partial_m A_{mn}=0$であるので, 非発散である. 
%%
\begin{equation}
  \tilde{v}_n = \epsilon_{nlm} \partial_l \psi_m
\end{equation}
%%
であることと\eqref{eq:pseudovel_AntiSymTensor_relation}を用いれば, 
反対称テンソル$A_{mn}$とスキュー速度$\tilde{\Dvect{v}}$の流線関数の間の関係は, 
%%
\begin{equation}
  A_{mn} = \epsilon_{mnp} \psi_p =
\begin{pmatrix}
 0 &\psi_3 &-\psi_2 \\
 -\psi_3 &0 &\psi_1 \\
 \psi_2  &-\psi_1 &0
\end{pmatrix}
\label{eq:AntiSymTensor_Psi_relation}
\end{equation}
%%
と得られる. 

\subsection{本節のまとめ}
%%
\begin{itemize}
 \item どのようなフラックスもスカラーの等値面を横切る成分(拡散フラックス)と沿う成分(スキューフラックス)に分割することができる.
 \item 拡散フラックス(たいてい勾配下向き)は, 対称拡散係数テンソルを用いた拡散によってもたらされる.
 \item スキューフラックスは, 反対称拡散係数テンソルを用いた拡散によってもたらされ, これはある非発散速度による移流と等価である.
 \item もし拡散係数が正であるならば, 拡散フラックスはトレサーの分散を減少させる(その場合, その拡散は勾配下向きである). 一方, スキュー拡散は分散に影響を及ぼさない.   
\end{itemize}
%%
次節で, これらの全てが, (大気や)海洋の大規模な流れにどのように関連しているかについて考えることにする. 