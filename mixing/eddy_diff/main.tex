%#BIBTEX jbibtex dynamics_description
%
%  水惑星設定における風成循環
%
%
%%%%%%%%%%%%%%%%%%%%%%%%%%%%%%%%%%%%%%%%%%%%%%%%%%%%%%%%
%%%%%%%%             Style  Setting             %%%%%%%%
% フォント: 12point (最大), 片面印刷
\documentclass[a4j,12pt,openbib,oneside]{jreport}

%%%%%%%%%%%%%%%%%%%%%%%%%%%%%%%%%%%%%%%%%%%%%%%%%%%%%%%%
%%%%%%%%             Package Include            %%%%%%%%
\usepackage{Dennou6}		% 電脳スタイル ver 6
\usepackage{ascmac}
\usepackage{tabularx}
\usepackage{graphicx}
\usepackage{amssymb}
\usepackage{amsmath}
\usepackage{mathrsfs}
\usepackage{framed}
\usepackage[round]{natbib}

%%%%%%%%%%%%%%%%%%%%%%%%%%%%%%%%%%%%%%%%%%%%%%%%%%%%%%%%
%%%%%%%%            PageStyle Setting           %%%%%%%%
\pagestyle{DAmyheadings}


%%%%%%%%%%%%%%%%%%%%%%%%%%%%%%%%%%%%%%%%%%%%%%%%%%%%%%%%
%%%%%%%%        Title and Auther Setting        %%%%%%%%
%%
%%  [ ] はヘッダに書き出される.
%%  { } は表題 (\maketitle) に書き出される.

\Dtitle{海洋における渦拡散}   % 変更不可
\Dauthor{河合 佑太}            % ゼミ担当者の名前
\Ddate{\the\year/\the\month/\the\day}        % ゼミの日時 (毎回変更すること)
\Dfile{main.tex}

%%%%%%%%%%%%%%%%%%%%%%%%%%%%%%%%%%%%%%%%%%%%%%%%%%%%%%%%
%%%%%%%%   Set Counter (chapter, section etc. ) %%%%%%%%
\setcounter{chapter}{0}    % 章番号
\setcounter{section}{0}    % 節番号
\setcounter{equation}{0}   % 式番号
\setcounter{page}{1}     % 必ず開始ページは明記する
\setcounter{figure}{0}     % 図番号
\setcounter{table}{0}      % 表番号
%\setcounter{footnote}{0}


%%%%%%%%%%%%%%%%%%%%%%%%%%%%%%%%%%%%%%%%%%%%%%%%%%%%%%%%
%%%%%%%%        Counter Output Format           %%%%%%%%
\def\thechapter{\arabic{chapter}}
\def\thesection{\arabic{chapter}.\arabic{section}}
\def\thesubsection{\arabic{chapter}.\arabic{section}.\arabic{subsection}}
\def\theequation{\arabic{chapter}.\arabic{section}.\arabic{equation}}
\def\thepage{\arabic{page}}
\def\thefigure{\arabic{chapter}.\arabic{section}.\arabic{figure}}
\def\thetable{\arabic{chapter}.\arabic{section}.\arabic{table}}
\def\thefootnote{*\arabic{footnote}}


%%%%%%%%%%%%%%%%%%%%%%%%%%%%%%%%%%%%%%%%%%%%%%%%%%%%%%%%
%%%%%%%%        Dennou-Style Definition         %%%%%%%%

%% 改段落時の空行設定
\Dparskip      % 改段落時に一行空行を入れる
%\Dnoparskip    % 改段落時に一行空行を入れない

%% 改段落時のインデント設定
\Dparindent    % 改段落時にインデントする
%\Dnoparindent  % 改段落時にインデントしない

%%%%%%%%%%%%%%%%%%%%%%%%%%%%%%%%%%%%%%%%%%%%%%%%%%%%%%%%
%%%%%%%%             Text Start                 %%%%%%%%
\begin{document}

\tableofcontents

\chapter{海洋における渦混合}
\subsection{準備}
今, 
%%
\begin{equation}
  \DD{\phi}{t} = \nabla \cdot (\kappa_m \nabla \phi)
\end{equation}
%%
なる方程式に従う, トレサー$\phi$を考える. 
ここで, $\kappa_m$は分子拡散係数である. 
もし移流速度が非発散であるならば, アンサンブル平均した方程式は, 
分子拡散を無視するとき, 
%%
\begin{equation}
  \DD{\overline{\phi}}{t} = - \nabla \cdot \overline{\Dvect{v}' \phi'}
\end{equation}
%%
となる. 
渦輸送を拡散によってパラメータ化するならば, そのとき, 
%%
\begin{equation}
  \Dvect{F} = \overline{\Dvect{v}' \phi'}
            = - \Dvect{K} \nabla \phi
\end{equation}
%%
と書ける. 
ここで, $\Dvect{K}$は一般に二階テンソルである. 
もし渦による項を拡散によってパラメータ化するならば, 
生じる問題は次の二点である. 
\begin{itemize}
 \item 渦拡散係数の\textbf{大きさ}. 可能性としては, 平均流の関数である. 
 \item 拡散係数テンソルの\textbf{構造}. 特に, 拡散テンソルの対称部分と反対称部分の分離構造. 
\end{itemize}
%%

\subsection{渦拡散係数の大きさ}

\subsection{拡散係数テンソルの構造: 対称輸送テンソル}
勾配方向に沿った拡散は, 対称輸送テンソルによる輸送である. 
今, 東西に一様な渦の統計的性質をもつ, 周期的なチャネルにおける輸送を考えよう. 
よって, 平均操作は東西平均である. 
このとき, トレサーの南北輸送, 鉛直輸送の両方に関心がある. 
%%
\begin{equation}
  \overline{v' \phi'} = - \kappa^{vy} \DP{\overline{\phi}}{y}
                        - \kappa^{vz} \DP{\overline{\phi}}{z}, 
\end{equation}
\begin{equation}
  \overline{w' \phi'} = - \kappa^{wy} \DP{\overline{\phi}}{y}
                        - \kappa^{wz} \DP{\overline{\phi}}{z}. 
\end{equation}
%%
ここで, 対称テンソルの仮定により$\kappa^{wy} = \kappa^{vz}$である. 
さまざまな輸送係数の関係は, 等密度面あるいは等エントロピー面の傾斜や
等エントロピー面に対する渦の軌跡の関係の両方によって影響を受けるだろう. 
カーテシアン座標($y-z$座標)において, 輸送テンソルは対角的である必要はないが, 
局所的には拡散係数テンソルが対角的となる自然座標系が必ず存在する. 
そのような座標系において, 拡散係数テンソル$\Dvect{S'}$は, 
%%
\begin{equation}
  \Dvect{S}' = \kappa_s
\begin{pmatrix}
    1 &0  \\
    0 &\alpha
  \end{pmatrix}
\end{equation}
%%
と書ける. 
ここで, $\kappa_S$は全体の大きさを決定する. 
また, $\alpha$は直交する二方向の輸送係数の比である. 
大規模な傾圧渦における流体の変位はほとんど水平であるが, 厳密には水平ではない. 
例えば, 変位は等密度面に沿うか, あるいは水平面と等密度面の間の角度にある. 
拡散係数テンソルが対角的である座標系は, 流体の変位が生じる方向に
沿った面によって定義される座標系であると言っても良い. 
流体の経路に沿う方向の輸送と垂直な方向の輸送はそれぞれ異なる物理現象の結果であるため, 
この主張は賢明である. 
したがって, 輸送テンソルがこの座標系において対角的であることを期待してよい. 

渦による変位は主に水平方向であるので, この座標系の傾斜は水平面に対して小さな角度で傾斜している. 
すなわち, $s = \tan \theta \approx \theta \ll 1$. 
さらに, パラメータ$\alpha$は小さい($\alpha \ll 1$)と思ってよい. 
なぜならば, $\alpha$は渦による流体の運動と垂直な方向の輸送を表現するからである. 
今, $y-z$座標系に変換するために, テンソル$\Dvect{S}$を角度$\theta$だけ回転させる. 
つまり, 
%%
\begin{eqnarray}
  \Dvect{S} &=& \kappa_S
\begin{pmatrix}
  \cos\theta &-\sin\theta \\
  \sin\theta &\cos\theta
\end{pmatrix}
\begin{pmatrix}
  1 &0 \\
  0 &\alpha
\end{pmatrix}
\begin{pmatrix}
  \cos\theta &\sin\theta \\
  -\sin\theta &\cos\theta
\end{pmatrix}
\\
 &\approx& \kappa_S
\begin{pmatrix}
 1+s^2 \alpha &s(1-\alpha) \\
 s(1-\alpha)  &s^2 + \alpha
\end{pmatrix} 
\\
 &\approx& \kappa_S
\begin{pmatrix}
  1 &s \\
  s &s^2+\alpha
\end{pmatrix}
\end{eqnarray}%
%
を得る. 
ただし, 二段目において$s \ll 1$であること, 
三段目においてさらに$\alpha \ll 1$であることを用いて近似を行った. 
三次元においても同様の方法をとればよい. 
もし渦輸送が渦による変位の面において等方的であれば, 三次元の輸送テンソルは, 
%%
\begin{equation}
  \Dvect{S}' = \kappa_s
\begin{pmatrix}
    1 &0 &0 \\
    0 &1 &0 \\
    0 &0 &\alpha 
  \end{pmatrix}
\end{equation}
%%
である. 
運動の傾斜は二次元ベクトル$\Dvect{s} = (s_x,s_y)$である. 
輸送テンソルを物理空間へと回転させれば, 二次元の場合と同様に, 
%%
\begin{eqnarray}
\Dvect{S} &=& \kappa_S 
\begin{pmatrix}
  1+s_y^2 + \alpha s_x^2 &(\alpha-1) s_x s_y   &(1-\alpha)s_x \\
  (\alpha-1) s_x s_y     &1+s_x^2+\alpha s_y^2 &(1-\alpha)s_y \\
  (1-\alpha)s_x          &(1-\alpha)s_y        &a + s^2
\end{pmatrix} 
\\
 &\approx& \kappa_S
\begin{pmatrix}
  1     &0       &s_x \\
  0     &1       &s_y \\
  s_x   &s_y     &a + s^2
\end{pmatrix} 
\end{eqnarray}
を得る. 
ここで, 二段目において$s,\alpha \ll 1$であることを用いて近似を行った. 
また, $s=s_x^2 + s_y^2$である. 


\subsection{拡散係数テンソルの構造: 反対称輸送テンソル}
反対称輸送テンソルはスキューフラックス(勾配ベクトルの垂直方向のフラックス)あるいは
偽の移流を発生させる. 
二次元(水平-鉛直面)において, 反対称輸送テンソルは, 速やかに, 
%%
\begin{equation}
  \Dvect{A} =
\begin{pmatrix}
  0 &-\kappa'_a \\
  \kappa'_a &0
\end{pmatrix}
\end{equation}
と書ける. 
ここで, $\kappa'_a$は空間・時間的に変化して良く, 流れ自体に依存する. 
また, 輸送の全体的な強さを決定する. 
推測により, 三次元においては, 
%%
\begin{equation}
  \Dvect{A} =
\begin{pmatrix}
  0 &0 &-\kappa_a^{'x} \\
  0 &0 &-\kappa_a^{'y} \\
  \kappa_a^{'x} &\kappa_a^{'y} &0
\end{pmatrix}
\end{equation}
と書ける. 
ここで, 上付き添字は成分を表す. 
今, 都合上ゲージを選択し, $A_{21}=-A_{12}=0$とした. 
残る選択は, 輸送係数の符号と大きさを決定することである. 


\subsection*{断熱的かつポテンシャルエネルギーを減少する渦輸送スキーム}
Gent と McWilliams は, 傾圧渦のフラックスによるトレサーの輸送のための, 
海洋モデルにおけるパラメタリゼーションを示した. 
これは, GM スキームとして知られる. 
以下の二つの性質の保持が, 彼らのスキームの基礎である. 
%%
\begin{itemize}
 \item トレーサのモーメントは保存されなければならない. 特に, 等密度面間の流体の量は保存されなければならない. これは, スキームが密度の勾配を横切って浮力を拡散させないことを示唆する. 
 \item 流れの有効位置エネルギーの量は減少しなければならない. この意味に置いて, このパラメタリゼーションは, 有効位置エネルギーを運動エネルギーに輸送する傾圧不安定の効果を模倣する. 
\end{itemize}
%%
一番目は, 反対称拡散係数テンソルを使うことによって自動的に満たされる. 
二つ目の性質は, 輸送係数を等密度面の傾斜に比例するように選択することによって満たされる. 
その場合には, 
%%
\begin{equation}
  \Dvect{A} = \kappa_a
\begin{pmatrix}
 0 &0 &-s_x \\
 0 &0 &-s_y \\
 s_x &s_y &0
\end{pmatrix}
\label{eq:GM90_AntiSymTensor}
\end{equation}
と書ける. 
ここで, $\Dvect{s}=(s_x, s_y)=\nabla_\rho z=- \nabla_z \rho/(\partial \rho/\partial z)$
は等密度面の傾斜である. 
温度場と塩分場を別々にもつ海洋モデルでは, \eqref{eq:GM90_AntiSymTensor}はそれらそれぞれに適用されるだろう. 
そのとき, 等密度面の傾斜は, 状態方程式を使って決定される. 
この輸送によって示唆される特徴が何かをより簡単に見るために, 塩分のない特別な場合を考えることにしよう. 
このとき, 浮力$b$は熱力学変数のみに依存する. 
等密度面の傾斜は$s=-(b_x/b_z, b_y/b_z)$, 水平渦輸送$\Dvect{F}_h=(F_x, F_y)$は, 
%%
\begin{equation}
 \Dvect{F}_h = -\left(-\kappa_a \Dvect{s} \DP{b}{z} \right)
             = - \kappa \nabla_z b
\end{equation}
%%
によって与えられる. 
正の$\kappa_a$に対して, これは従来の勾配を下る(downgradient)拡散である. 

一方, 鉛直方向の輸送は, 
%%
\begin{equation}
  F_z = - \kappa_a \left(s_x\DP{b}{x} + s_y\DP{b}{y}\right)
      = \kappa_a s^2 \DP{b}{z} 
\end{equation}
%%
によって与えられる. 
ここで, $s^2=\Dvect{s}\cdot\Dvect{s}$. 
このフラックスは勾配を上る向きである(upgradient). 
しかしながら, 合計のスキューフラックスは勾配を下る向きでも上る向きでもない. 

勾配を下る向きの水平フラックスと勾配を上る方向の鉛直フラックスの組み合わせは, 
それぞれの密度間隔内の流体の体積を保存すると同時に, 流れの位置エネルギーを減少させるように
作用する. 
鉛直方向の勾配上向きフラックスは, 有効位置エネルギーを減少させるための必要性の結果である. 
暖かく軽い流体が冷たく重い流体の上に乗っている, 静的安定な状況を考えると, 
勾配下向きの鉛直拡散は流体の重心を上昇させる. 結果, ポテンシャルエネルギーは増加する. 
これは, 傾圧不安定の作用と逆である. 
よって, 鉛直拡散の符号は負でなければならず, 
\eqref{eq:GM90_AntiSymTensor}の構造(よって正の水平拡散係数)と共にこのことは, 
上述した二つの性質の両方を満たすことを可能にする. 
このパラメタリゼーションは, 全エネルギーを保存しない. 
すなわち, ポテンシャルエネルギーは対応する運動エネルギーの増加によってバランスされない. 
むしろ, 散逸によって失われることが仮定される. 
最後に, (スキュー)渦拡散係数の大きさは, 前述の現象論的推定によって決定される. 

\subsection*{渦輸送速度}
%%
よって, 渦輸送速度は, 
\begin{equation}
 \begin{split}
   \Dvect{\tilde{u}} &= - \DP{}{z} (\kappa_a \Dvect{s}), \\
   \tilde{w} &= \nabla_z \cdot (\kappa_a \Dvect{s})
 \end{split}
\end{equation}
%%
と与えられる%
\footnote{
スキュー速度$\Dvect{\tilde{v}}$と反対称拡散係数テンソル$\Dvect{A}$の関係は, 
%%
\begin{equation*}
  \tilde{v}_n = - \nabla_m A_{mn}
\end{equation*}
%%
によって与えられる. 
}. 
また, $\Dvect{A}$と関係する流線関数は, 
%%
\begin{equation}
  \Dvect{\psi} = (-\kappa_a s_y, \kappa_a s_x, 0) = \Dvect{k} \times \kappa_a \Dvect{s}
\end{equation}
%%
によって与えられる%
\footnote{
流線関数と反対称拡散係数テンソル$\Dvect{A}$の関係は, 
$\tilde{v}_n = \epsilon_{lmn}\nabla_l \psi_m$および$\tilde{v}_n = - \nabla_m A_{mn}$より, 
$$
A_{mn} = \epsilon_{mnp} \psi_p
$$
と与えられる. 
}. 

Gent-McWilliams のパラメタリゼーションの実装には, 二つの等価な方法があり, 
一つはスキューフラックスを用い, もう一方は渦輸送速度による移流を使う. 
前者は, 渦フラックスを
\begin{equation}
  \Dvect{F}_{\rm sk} = - \nabla b \times \Dvect{\psi}
\end{equation}
と書く. 
一方, 後者は, 
\begin{equation}
  \Dvect{F}_{\rm ad} = - b \nabla \times \Dvect{\psi}
\end{equation}
と書く. 
発散をとれば, 両者は等価な表現となることに注意されたい. 


\newpage



%%

%% 参考文献
\bibliographystyle{plainnat}
\bibliography{./../../Dennou-OGCM_reflist}

\end{document}
%%%%%%%%              Text End                  %%%%%%%%
%%%%%%%%%%%%%%%%%%%%%%%%%%%%%%%%%%%%%%%%%%%%%%%%%%%%%%%%
