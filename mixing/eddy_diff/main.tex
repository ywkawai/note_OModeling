%#BIBTEX bibtex main
%
%  水惑星設定における風成循環
%
%
%%%%%%%%%%%%%%%%%%%%%%%%%%%%%%%%%%%%%%%%%%%%%%%%%%%%%%%%
%%%%%%%%             Style  Setting             %%%%%%%%
% フォント: 12point (最大), 片面印刷
\documentclass[a4j,12pt,openbib,oneside]{jreport}

%%%%%%%%%%%%%%%%%%%%%%%%%%%%%%%%%%%%%%%%%%%%%%%%%%%%%%%%
%%%%%%%%             Package Include            %%%%%%%%
\usepackage[dvipdfmx]{graphicx}
\usepackage{Dennou6}		% 電脳スタイル ver 6
\usepackage{ascmac}
\usepackage{tabularx}
\usepackage{amssymb}
\usepackage{amsmath}
\usepackage{mathrsfs}
\usepackage{framed}
\usepackage[round]{natbib}

%%%%%%%%%%%%%%%%%%%%%%%%%%%%%%%%%%%%%%%%%%%%%%%%%%%%%%%%
%%%%%%%%            PageStyle Setting           %%%%%%%%
\pagestyle{DAmyheadings}


%%%%%%%%%%%%%%%%%%%%%%%%%%%%%%%%%%%%%%%%%%%%%%%%%%%%%%%%
%%%%%%%%        Title and Auther Setting        %%%%%%%%
%%
%%  [ ] はヘッダに書き出される.
%%  { } は表題 (\maketitle) に書き出される.

\Dtitle{海洋における傾圧渦の輸送の表現}   % 変更不可
\Dauthor{河合 佑太}            % ゼミ担当者の名前
\Ddate{\the\year/\the\month/\the\day}        % ゼミの日時 (毎回変更すること)
\Dfile{main.tex}

%%%%%%%%%%%%%%%%%%%%%%%%%%%%%%%%%%%%%%%%%%%%%%%%%%%%%%%%
%%%%%%%%   Set Counter (chapter, section etc. ) %%%%%%%%
\setcounter{chapter}{0}    % 章番号
\setcounter{section}{0}    % 節番号
\setcounter{equation}{0}   % 式番号
\setcounter{page}{1}     % 必ず開始ページは明記する
\setcounter{figure}{0}     % 図番号
\setcounter{table}{0}      % 表番号
%\setcounter{footnote}{0}


%%%%%%%%%%%%%%%%%%%%%%%%%%%%%%%%%%%%%%%%%%%%%%%%%%%%%%%%
%%%%%%%%        Counter Output Format           %%%%%%%%
\def\thechapter{\arabic{chapter}}
\def\thesection{\arabic{chapter}.\arabic{section}}
\def\thesubsection{\arabic{chapter}.\arabic{section}.\arabic{subsection}}
\def\theequation{\arabic{chapter}.\arabic{section}.\arabic{equation}}
\def\thepage{\arabic{page}}
\def\thefigure{\arabic{chapter}.\arabic{section}.\arabic{figure}}
\def\thetable{\arabic{chapter}.\arabic{section}.\arabic{table}}
\def\thefootnote{*\arabic{footnote}}


%%%%%%%%%%%%%%%%%%%%%%%%%%%%%%%%%%%%%%%%%%%%%%%%%%%%%%%%
%%%%%%%%        Dennou-Style Definition         %%%%%%%%

%% 改段落時の空行設定
\Dparskip      % 改段落時に一行空行を入れる
%\Dnoparskip    % 改段落時に一行空行を入れない

%% 改段落時のインデント設定
\Dparindent    % 改段落時にインデントする
%\Dnoparindent  % 改段落時にインデントしない

%%%%%%%%%%%%%%%%%%%%%%%%%%%%%%%%%%%%%%%%%%%%%%%%%%%%%%%%
%%%%%%%%             Text Start                 %%%%%%%%
\begin{document}

\tableofcontents

\chapter{海洋における傾圧渦の輸送の表現}

%%%%%%%%%%
%%
\section{はじめに}
%%
海洋大循環の数値実験の中では, 特定の格子点において静的不安定(ポテンシャル密度の鉛直勾配が正)が発生し得る.
自然では, 対流過程が海水カラムの静的不安定を速やかに解消する. 
しかし, 静水圧平衡の仮定を行う静水圧海洋大循環モデルでは対流過程は表現できないため,
この過程をパラメーター化しなければならない.
海洋大循環モデルにおける対流過程のパラメーター化には, 大きくは以下のようなものがある.
%%
\begin{itemize}
 \item 非貫入型対流調節
 \item 貫入型対流調節
 \item 乱流クロージャースキーム      
\end{itemize}
%%
本ノートでは, 海洋大循環モデルにおける対流過程のパラメタリゼーションの内,
非貫入型対流調節について述べる.


\section{拡散係数テンソル}
%%
分子拡散の効果を除いて自由に時間発展するトレサーは, 
%%
\begin{equation}
  \DD{\phi}{t} = \nabla \cdot (\kappa_m \nabla \phi))
\label{eq:tracer_eq_molecular_diff}
\end{equation}
%%
に従う. 
ここで, $\kappa_m$は分子拡散係数であり, 正のスカラー量である. 
より一般的には, 
%%
\begin{equation}
  \DD{\phi}{t} = - \nabla \cdot \Dvect{F} = \nabla \cdot \Dvect{K} \nabla \phi
\label{eq:tracer_eq_general_diff}
\end{equation}
%%
と書かれる. 
ここで, $\Dvect{K}$は($\phi$がスカラーであれば)2階テンソルであり, 
$\Dvect{F}=-\Dvect{K}\nabla \phi$は$\phi$の拡散フラックスである. 
そのフラックスは, $\phi$の等値面を横切る成分(拡散フラックスと呼ばれる)と, 
沿う成分(スキューフラックスと呼ばれる)をもつ. 
これらのフラックスはそれぞれ, 拡散係数フラックスの対称成分$\Dvect{K}$と反対称成分$\Dvect{A}$に関係している. 
ここで, 
%%
\begin{equation}
  \Dvect{K} = \Dvect{S} + \Dvect{A}
\end{equation}
%%
であり, 成分表示を用いて, 
%%
\begin{equation}
 S_{mn} = \dfrac{1}{2} (K_{mn} + K_{nm}), \;\;\;
 A_{mn} = \dfrac{1}{2} (K_{mn} - K_{nm}) 
\end{equation}
%% 
である. 
反対称テンソルの対角要素は, ゼロである. 
これらの二種類のテンソルによってもたらされる輸送は, 異なる物理的特徴をもつことを以下で述べる. 

\subsection{対称テンソルによる輸送(拡散)}

\subsubsection*{最も簡単な例}
最も簡単な場合は媒質が等方的な場合であり, このとき$\Dvect{K}$は対角的かつ要素の値は同じである. 
すなわち, 
%%
\begin{equation}
  \Dvect{K} = \Dvect{S}
\begin{pmatrix}
 \kappa &0 &0 \\
 0 &\kappa &0 \\
 0 &0 &\kappa
\end{pmatrix}
\end{equation}
%%
と書かれ, よく知られた$\Dvect{F}=-\kappa\nabla \phi$を得る. 
したがって, \eqref{eq:tracer_eq_general_diff}は\eqref{eq:tracer_eq_molecular_diff}と同じ形式をもつ. 
もし$\kappa$が正ならば, そのときフラックスはたとえ$\kappa$が空間非一様であっても, 
\textbf{勾配下向き}(downgradient)である. すなわち, 
%%
\begin{equation}
  \Dvect{F} \cdot \nabla \phi \le 0. 
\end{equation}
%%
さらに, そのような拡散は分散が散逸する. 
それを確かめるために, 運動方程式
%%
\begin{equation}
 \DP{\phi}{t} = \nabla \cdot (\kappa \nabla \phi)
\end{equation}
%%
を考えよう. 
両辺に$\phi$を掛けて領域$V$に渡って積分し, 部分積分を実行すると, 
%%
\begin{equation}
   \dfrac{1}{2} \DD{}{t} \int_V \phi^2 \Dd{V}
 = \int_V \int_{V} \Dvect{F} \cdot \nabla \phi \Dd{V}
 = - \kappa (\nabla \phi)^2 \le 0
\end{equation}
%%
を得る. 
ここで, $\phi$物質の境界を横切るフラックスは存在しないことを仮定した. 
しかし, 拡散は場の一次モーメントを保存する. 
すなわち, 
%%
\begin{equation}
  \DD{}{t} \int \phi \Dd{V} = \int_V \nabla \cdot (\kappa \nabla \phi) \Dd{V} = 0 
\end{equation}
%%
である. 
ここで, 再び境界を横切るフラックスはゼロであることを仮定した. 

\subsubsection*{一般化}
対称な拡散係数テンソルがもたらす輸送は, 拡散フラックス
%%
\begin{equation}
  \Dvect{F}_d = - \Dvect{S} \nabla \phi
  = -S_{mn} \partial_n \phi
\end{equation}
%%
である. 
ここで, 添え字に対してアインシュタイン縮約規則を適用するものとする. 
一般的に, フラックス$\Dvect{F}_d$はトレサーの勾配と平行な成分をもつ. 
簡単化のために, 
%%
\begin{equation}
 \DP{\phi}{t} = - \nabla \cdot \Dvect{F}_d 
     = \nabla \cdot (\Dvect{S} \nabla \phi)
\end{equation}
%%
なる運動方程式を考えよう. 
この方程式は, 境界を横切るフラックスが存在しなければ, $\phi$の一次モーメントを保存する. 
トレサーの分散は, 
%%
\begin{equation}
   \dfrac{1}{2} \DD{}{t} \int_V \phi^2 \Dd{V}
 = \int_{V} \phi \nabla \cdot ( \Dvect{S} \nabla \phi) \Dd{V}
 = - \int_V (\Dvect{S}\nabla \phi) \cdot \nabla \phi \Dd{V}
\end{equation}
%%
によって時間発展する. 
これは, $\Dvect{S}$が半正定値, すなわち
%%
\begin{equation}
  \nabla \phi \Dvect{S} \nabla \phi = \partial_m \phi S_{mn} \partial_n \phi \ge 0
\end{equation}
%%
であるならば, 右辺は負もしくはゼロであることが示される. 
またこのとき, そのような拡散がもたらすフラックスは, 
%%
\begin{equation}
  \Dvect{F}_d \cdot \nabla \phi = -\Dvect{S}\nabla\phi \cdot \nabla \phi \le 0
\end{equation}
という意味で勾配下向きである. 

\subsection{スキューフラックスによる輸送}
反対称テンソルと関係した輸送は, $\phi$の勾配と直交する. 
すなわち, 勾配上向きでも勾配下向きでもない. 
そのフラックスは, 
%%
\begin{equation}
  \Dvect{F}_{sk} = - \Dvect{A} \nabla \phi
  = - A_{mn} \partial_n \phi
\label{eq:skew_flux1}
\end{equation}
%%
であり, 
よって, 
%%
\begin{equation}
  \Dvect{F}_{sk} \cdot \nabla \phi
  = - \Dvect{A} \nabla \phi \cdot \nabla \phi = - A_{mn} \partial_n \phi \partial_m \phi = 0
\end{equation}
%%
である. 
ここで, 最後の等号において, $\Dvect{A}$の反対称性(対称テンソルと反対称テンソルの縮約はゼロ)を用いた%
\footnote{
$A_{mn} (\partial_n \phi \partial_m)$において, カッコ内が対称テンソルであることに注意して$S_{mn}$とおけば, 
$$
A_{mn} S_{mn} = - A_{nm} S_{nm} = - A_{mn} S_{mn}
$$
となり, $A_{mn} S_{mn}=0$を得る. 
}.
この理由により, 反対称テンソルと関係した輸送は, スキュー\textbf{フラックス}やスキュー\textbf{拡散}と呼ばれる. 
前者はトレサーの勾配に垂直なフラックスに一般的に用いられる用語であり, 
後者はスキューフラックスが反対称的な拡散係数を用いてパラメータ化されるときに用いられる用語である. 
このことから, 今トレサーが
%%
\begin{equation}
  \DP{\phi}{t} = \nabla \cdot (\Dvect{A} \nabla \phi)
\end{equation}
%%
に従うならば, 両辺に$\phi$を掛け部分積分を実行すれば, トレサーの分散が保存することが示される. 
ただし, 境界でフラックスがゼロであることを仮定した. 
つまり, \textbf{スキュー拡散は変数の分散に影響を及ぼさない}. 
非発散流による移流は, この性質を同じようにもつ物理的過程である. 
スキュー拡散はそのような移流と物理的に等価であり, そ
こではスキュー拡散フラックスの発散は, 適切に選択された移流フラックスの発散と同じである. 
そのことを以下で述べる. 

今, 
%%
\begin{equation}
  \Dvect{F}_{ad} \equiv \tilde{\Dvect{v}} \phi
\end{equation}
%%
なる形式のフラックスを, トレサー$\phi$の移流フラックスとして定義する. 
ここで, $\tilde{\Dvect{v}}$は非発散なベクトル場である. 
このフラックスの発散は, 単に, 
%%
\begin{equation}
  \nabla \cdot \Dvect{F}_{ad} = \tilde{\Dvect{v}} \cdot \nabla \phi
\end{equation}
%%
である. 
場$\tilde{\Dvect{v}}$は\textbf{擬速度}あるいは\textbf{準速度}と呼ばれるだろう. 
それは速度のように作用するが, どの流体粒子の速度である必要は無い. 
$\tilde{\Dvect{v}}$は非発散であるので, 
%%
\begin{equation}
 \tilde{\Dvect{v}} = \nabla \times \Dvect{\psi}
\end{equation}
%%
となるような, ベクトルの流線関数$\Dvect{\psi}$を定義することができる. 
場$\Dvect{\psi}$は一意ではない. 
任意関数の勾配を$\Dvect{\psi}$に付け加えてもよい. 
つまり, もし$\Dvect{\psi}' = \Dvect{\psi} + \nabla \gamma$であるならば, 
このとき, $\tilde{\Dvect{v}}=\nabla \times \Dvect{\psi} = \nabla \times \Dvect{\psi}'$となるからである.
スカラー場$\gamma$は\textbf{ゲージ}と呼ばれ, その選択の自由度はゲージの自由度である. 

移流フラックス$\Dvect{F}_{ad}$は, スキューフラックス$\Dvect{F}_{sk}$と以下の式によって関係付けられる. 
%%
\begin{equation}
  \phi \tilde{\Dvect{v}} 
 = \phi \nabla \times \Dvect{\psi} 
 = \nabla \times (\phi \Dvect{\psi}) - \nabla \phi \times \Dvect{\psi}, 
\end{equation}
%%
もしくは, 
%%
\begin{equation}
  \Dvect{F}_{ad} = \Dvect{F}_{r} +\Dvect{F}_{sk}. 
\end{equation}
%%
ここで, $\Dvect{F}_r=\nabla \times (\phi \Dvect{\psi})$は, 発散を伴わない回転のフラックスである.  
また, 
%%
\begin{equation}
  \Dvect{F}_{sk} = - \nabla \phi \times \Dvect{\psi}
\label{eq:skew_flux2}
\end{equation}
%%
は, スキューフラックスである. 
$\nabla \cdot \Dvect{F}_r=0$であるので, 
%%
\begin{equation}
  \nabla \cdot \Dvect{F}_{ad} = \nabla \cdot \Dvect{F}_{sk}
\end{equation}
%%
である. 
スキューフラックス$-\nabla \phi \times \Dvect{\psi}$と移流フラックス$\phi \nabla \times \Dvect{\psi}$は, 
一般的には異なる大きさや向きを持つことに注意する必要がある. 
等しいのはそれらの発散のみである. 
もし\eqref{eq:skew_flux1},\eqref{eq:skew_flux2}で与えられるスキューフラックスの発散が同じであるならば, 
そのとき$\Dvect{\psi}$は反対称テンソル$\Dvect{A}$と関係付けられなければならない. 
\eqref{eq:skew_flux1}を用いて, 
%%
\begin{equation}
\begin{split}
   \nabla \cdot \Dvect{F}_{sk} 
   &= - \partial_m ( A_{mn} \partial_n \phi) \\
   &= - (\partial_n \phi) (\partial_m A_{mn}) - [A_{mn} \partial_n \partial_m \phi] \\
   &= - \partial_n (\phi \partial_m A_{mn}) + [\phi \partial_n \partial_m A_{mn}]
\end{split}
\end{equation}
%%
を得る. 
ここで, 角括弧内の量は$\Dvect{A}$の反対称性によりゼロである. 
しかし, スキューフラックスの発散は移流フラックスの発散と等しい. 
すなわち, 
%%
\begin{equation}
  \nabla \cdot \Dvect{F}_{sk} 
 = \nabla \cdot \Dvect{F}_{ad} 
 = \partial_n (\phi \tilde{v}_n)
\end{equation}
%%
であるので, $\Dvect{A}$と関係するスキュー速度は, 
%%
\begin{equation}
  \tilde{v}_n = - \partial_m A_{mn}
\label{eq:pseudovel_AntiSymTensor_relation}
\end{equation}
%%
によって与えられる. 
この速度は, $\partial_n \partial_m A_{mn}=0$であるので, 非発散である. 
%%
\begin{equation}
  \tilde{v}_n = \epsilon_{nlm} \partial_l \psi_m
\end{equation}
%%
であることと\eqref{eq:pseudovel_AntiSymTensor_relation}を用いれば, 
反対称テンソル$A_{mn}$とスキュー速度$\tilde{\Dvect{v}}$の流線関数の間の関係は, 
%%
\begin{equation}
  A_{mn} = \epsilon_{mnp} \psi_p =
\begin{pmatrix}
 0 &\psi_3 &-\psi_2 \\
 -\psi_3 &0 &\psi_1 \\
 \psi_2  &-\psi_1 &0
\end{pmatrix}
\label{eq:AntiSymTensor_Psi_relation}
\end{equation}
%%
と得られる. 

\subsection{本節のまとめ}
%%
\begin{itemize}
 \item どのようなフラックスもスカラーの等値面を横切る成分(拡散フラックス)と沿う成分(スキューフラックス)に分割することができる.
 \item 拡散フラックス(たいてい勾配下向き)は, 対称拡散係数テンソルを用いた拡散によってもたらされる.
 \item スキューフラックスは, 反対称拡散係数テンソルを用いた拡散によってもたらされ, これはある非発散速度による移流と等価である.
 \item もし拡散係数が正であるならば, 拡散フラックスはトレサーの分散を減少させる(その場合, その拡散は勾配下向きである). 一方, スキュー拡散は分散に影響を及ぼさない.   
\end{itemize}
%%
次節で, これらの全てが, (大気や)海洋の大規模な流れにどのように関連しているかについて考えることにする. 

%%%%%%%%%%%
\section{海洋における渦拡散}
\subsection{準備}
今, 
%%
\begin{equation}
  \DD{\phi}{t} = \nabla \cdot (\kappa_m \nabla \phi)
\end{equation}
%%
なる方程式に従う, トレサー$\phi$を考える. 
ここで, $\kappa_m$は分子拡散係数である. 
もし移流速度が非発散であるならば, アンサンブル平均した方程式は, 
分子拡散を無視するとき, 
%%
\begin{equation}
  \DD{\overline{\phi}}{t} = - \nabla \cdot \overline{\Dvect{v}' \phi'}
\end{equation}
%%
となる. 
渦輸送を拡散によってパラメータ化するならば, そのとき, 
%%
\begin{equation}
  \Dvect{F} = \overline{\Dvect{v}' \phi'}
            = - \Dvect{K} \nabla \phi
\end{equation}
%%
と書ける. 
ここで, $\Dvect{K}$は一般に二階テンソルである. 
もし渦による項を拡散によってパラメータ化するならば, 
生じる問題は次の二点である. 
\begin{itemize}
 \item 渦拡散係数の\textbf{大きさ}. 可能性としては, 平均流の関数である. 
 \item 拡散係数テンソルの\textbf{構造}. 特に, 拡散テンソルの対称部分と反対称部分の分離構造. 
\end{itemize}
%%



\subsection{渦拡散係数の大きさ}
今, トレサーの性質の南北輸送によるモーメントに注意を限定する. 
 %%
\begin{equation}
 \overline{v' \phi'}
 = - \kappa^{vy} \DP{\bar{\phi}}{y} - \kappa^{vz} \DP{\bar{\phi}}{z}. 
\end{equation}
%%
ここで, $\kappa^{vy}, \kappa^{vz}$は渦拡散係数テンソルの成分である.
これらの成分は, 長さ$\times$速度の次元をもつ. 
拡散が渦の運動を表現する限り, $\kappa^{vy}$の近似的な大きさは,
%%
\begin{equation}
  \kappa \sim v' l'
\end{equation}
%%
であることが期待される.
ここで, $v'$は渦の水平速度の典型的な大きさである.
また, $l'$は渦の\textbf{混合長}であり, 一般的に渦の典型的な長さスケールが取られる.
スケールが大きい渦ほど, また大きなエネルギーをもつ渦ほど, 平均流により大きな影響を与える.
$v', l'$は, 流れの条件に依存した, 多くの適切な方法により推定できる.
以下に, その推定方法の例を挙げる.
このとき, 成分$\kappa^{vz}$の大きさは, パーセルの変位の平面を選択することによって推定できる.

おそらく, 置かれるで最も簡単な仮定は, 渦は傾圧不安定の結果であると事実による.
よって, 渦の長さスケールは不安定のスケール, すなわち第一変形半径と考えて良い.
また, 渦の速度は平均流$\bar{u}$と同程度の近似的な大きさを持つと考えて良く,
よって,
%%
\begin{equation}
 \kappa^{vy} \sim L_d \bar{u} = \dfrac{NH \bar{u}}{f}
 \label{eq:eddyDiff_y_SimpleForm1}
\end{equation}
%%
を与える.
あるいは, 特徴的な渦の時間スケールを$T_e$とするとき, $\kappa^{vy} \sim l^2/T_e$において,
$T_e \sim L_d/\bar{u}, \; l' \sim L_d$と取るならば, 上の結果を導くことができる. 
\eqref{eq:eddyDiff_y_SimpleForm1}はまた,
%%
\begin{equation}
 \kappa^{vy} \sim L_d \bar{u} \sim \dfrac{L_d^2 f}{\sqrt{Ri}}
             \sim L_d^2 F_r f
\end{equation}
%%
と書かれることに注意が必要である. 
ここで, $R_i \equiv N^2/\Lambda^2 = N^2 H^2/\bar{u}^2$, $F_r\equiv U/(NH)$はそれぞれ,
本問題に対するリチャードソン数とフルード数である.

少しだけより一般的には, もしより大きなスケールへのカスケードが存在するならば,
そのとき渦のスケール$L_e$は変形半径よりも大きくなるだろう.
状況に応じて, $L_e$は, (渦が領域サイズまで成長するならば)領域のスケール,
($\beta$効果がカスケードを止めるならば)$\beta$スケール,
(もしかすると$\beta$と共に)摩擦の効果によって決定されるいくつかのスケールかもしれない. 
しかしながら, \citet{vallis2006atmospheric}の9.3節の議論は,
全ての場合で渦の時間スケールはEadyの時間スケールであることを示唆する.
故に,
%%
\begin{subequations} %12:00の式群
 \begin{equation}
   \textrm{渦の長さスケール:} \sim L_e
 \end{equation}
 \begin{equation}
   \textrm{渦の時間スケール:} \sim T_e \sim L_d/\bar{u}
 \end{equation}
 \begin{equation}
   \textrm{渦の速度スケール} U_e \sim \bar{u} (L_e/L_d)
 \end{equation}
 \label{eq:scaling_eddy_basedEadyTime}
\end{subequations}
%%
と取る.
これらは, 水平拡散係数の一般的な推定として,
%%
\begin{equation}
 \boxed{
  \kappa^{vy} \sim \bar{u} \left(\dfrac{L_e^2}{L_d}\right)
  }
 \label{eq:eddyDiff_y_SimpleForm2}
\end{equation}
%%
を与える.
\eqref{eq:eddyDiff_y_SimpleForm1}で与えられる推定は,
$L_e=L_d$としたときの上式による推定の特別な場合である.
よって, 渦のスケールが変形半径よりはるかに大きければ, 二種類の推定は異なるだろう.

逆カスケードがロスビー波によって修正される場合には, 摩擦の効果を無視すれば, 
渦のスケールは$\beta$スケールであると考える.
このとき, 渦のスケールは, \eqref{eq:scaling_eddy_basedEadyTime}の三式目を用いて, 
%%
\begin{equation}
 L_e \sim L_\beta = \left(\dfrac{U_e}{\beta}\right)^{1/2}
  = \dfrac{\bar{u}}{\beta L_d}
  \label{eq:eddyScale_SimpleForm2}
\end{equation}
%%
と取られる. 
渦の速度スケールは, \eqref{eq:scaling_eddy_basedEadyTime}の二式目を用いて,
%%
\begin{equation}
 U_e \sim \bar{u}\dfrac{L_e}{L_d} = \dfrac{\bar{u}^2}{\beta L_d^2}
  \label{eq:eddyVelScale_SimpleForm2}
\end{equation}
%%
である.
また, \eqref{eq:eddyScale_SimpleForm2}, \eqref{eq:eddyVelScale_SimpleForm2}を組み合わせれば,
渦拡散係数の推定として,
%%
\begin{equation}
 \kappa^{vy} \sim \dfrac{\bar{u}^3}{\beta^2 L_d^3}
 \label{eq:eddyDiffCoef_y_betaScale}
\end{equation}
%%
を与える.
エネルギーの逆カスケード率$\varepsilon$を使った同様の推定は, 
%%
\begin{equation}
  \kappa^{vy} \sim \left(\dfrac{\varepsilon^3}{\beta^4}\right)^{1/5}
\end{equation}
%%
と得られる. 
この表現は, $\kappa$を決定する唯一の要素が$\varepsilon$と$\beta$であると仮定するならば,
次元解析によって純粋に得られるだろう%
\footnote{
エネルギーの逆カスケード率の単位は, [J/(s$\cdot$kg)] である. 
}.
(例えば大気のエネルギー流量を計算することによって)もし$\varepsilon$が独立に既知であれば, この推定は役に立つ. 

\subsubsection*{まとめ}
渦拡散係数の大きさは, 渦の速度スケールとエネルギーを保持する長さスケールの積として推定される.
もし渦の時間スケールが Eady の時間スケールであることを仮定するならば, \eqref{eq:eddyDiff_y_SimpleForm2}を得る.
ここで, $L_e$は未決定である.
もし渦のスケールが$\beta$スケールであるならば, \eqref{eq:eddyDiff_y_SimpleForm2}は,
\eqref{eq:eddyDiffCoef_y_betaScale}となる.
しかし, 大気と海洋のどちらにおいても, $\beta$スケールは変形半径よりはるかに大きい(約10倍ほど)が,
問題を複雑にすることに逆カスケードは$\beta$スケールで止まる必要は無い%
\footnote{
簡単のために, 順圧渦度方程式
$$
\DP{\zeta}{t} + J(\psi, \zeta) + \beta \DP{\psi}{x} = F - r \zeta + \nu \nabla^2 \zeta
$$
を使って考えよう.
ここで, 粘性$\nu$は小さく, 小さなスケールにおいてエンストロフィーを取り除くためだけに働き,
エネルギーは取り除かない.
また, 強制$F$は系にエネルギーを注入する.
そのエネルギーは大スケールへとカスケードし, 線形の摩擦項$-r\zeta$によって取り除かれる.
もし摩擦項が十分に大きければ, エネルギーは$\beta$効果を感じる前に取り除かれるだろう. 
}
観測によれば, 海洋のいくつかの領域では, 逆カスケードに対する-5/3 べき乗則の事実がいくつか存在するが,
大気には存在しない.
他の海洋の領域では, 逆カスケードが組織化され得る前に, 渦は互いから離れ, 不安定領域から離れるように移流されるか, 
あるいはロスビー波によって分散される.
その場合, エネルギーは変形スケールに留まり続けるだろう.
これらの議論は, 実用的な推定は可能であるが,
大気海洋の両方において, 渦拡散係数の大きさを確信を持って決定できないことを示唆する. 


\subsection{拡散係数テンソルの構造: 対称輸送テンソル}
勾配方向に沿った拡散は, 対称輸送テンソルによる輸送である. 
今, 東西に一様な渦の統計的性質をもつ, 周期的なチャネルにおける輸送を考えよう. 
よって, 平均操作は東西平均である. 
このとき, トレサーの南北輸送, 鉛直輸送の両方に関心がある. 
%%
\begin{equation}
  \overline{v' \phi'} = - \kappa^{vy} \DP{\overline{\phi}}{y}
                        - \kappa^{vz} \DP{\overline{\phi}}{z}, 
\end{equation}
\begin{equation}
  \overline{w' \phi'} = - \kappa^{wy} \DP{\overline{\phi}}{y}
                        - \kappa^{wz} \DP{\overline{\phi}}{z}. 
\end{equation}
%%
ここで, 対称テンソルの仮定により$\kappa^{wy} = \kappa^{vz}$である. 
さまざまな輸送係数の関係は, 等密度面あるいは等エントロピー面の傾斜や
等エントロピー面に対する渦の軌跡の関係の両方によって影響を受けるだろう. 
カーテシアン座標($y-z$座標)において, 輸送テンソルは対角的である必要はないが, 
局所的には拡散係数テンソルが対角的となる自然座標系が必ず存在する. 
そのような座標系において, 拡散係数テンソル$\Dvect{S'}$は, 
%%
\begin{equation}
  \Dvect{S}' = \kappa_s
\begin{pmatrix}
    1 &0  \\
    0 &\alpha
  \end{pmatrix}
\end{equation}
%%
と書ける. 
ここで, $\kappa_S$は全体の大きさを決定する. 
また, $\alpha$は直交する二方向の輸送係数の比である. 
大規模な傾圧渦における流体の変位はほとんど水平であるが, 厳密には水平ではない. 
例えば, 変位は等密度面に沿うか, あるいは水平面と等密度面の間の角度にある. 
拡散係数テンソルが対角的である座標系は, 流体の変位が生じる方向に
沿った面によって定義される座標系であると言っても良い. 
流体の経路に沿う方向の輸送と垂直な方向の輸送はそれぞれ異なる物理現象の結果であるため, 
この主張は賢明である. 
したがって, 輸送テンソルがこの座標系において対角的であることを期待してよい. 

渦による変位は主に水平方向であるので, この座標系の傾斜は水平面に対して小さな角度で傾斜している. 
すなわち, $s = \tan \theta \approx \theta \ll 1$. 
さらに, パラメータ$\alpha$は小さい($\alpha \ll 1$)と思ってよい. 
なぜならば, $\alpha$は渦による流体の運動と垂直な方向の輸送を表現するからである. 
今, $y-z$座標系に変換するために, テンソル$\Dvect{S}$を角度$\theta$だけ回転させる. 
つまり, 
%%
\begin{eqnarray}
  \Dvect{S} &=& \kappa_S
\begin{pmatrix}
  \cos\theta &-\sin\theta \\
  \sin\theta &\cos\theta
\end{pmatrix}
\begin{pmatrix}
  1 &0 \\
  0 &\alpha
\end{pmatrix}
\begin{pmatrix}
  \cos\theta &\sin\theta \\
  -\sin\theta &\cos\theta
\end{pmatrix}
\\
 &\approx& \kappa_S
\begin{pmatrix}
 1+s^2 \alpha &s(1-\alpha) \\
 s(1-\alpha)  &s^2 + \alpha
\end{pmatrix} 
\\
 &\approx& \kappa_S
\begin{pmatrix}
  1 &s \\
  s &s^2+\alpha
\end{pmatrix}
\label{eq:EddDiffSymTensor2D}
\end{eqnarray}%
%
を得る. 
ただし, 二段目において$s \ll 1$であること, 
三段目においてさらに$\alpha \ll 1$であることを用いて近似を行った. 
三次元においても同様の方法をとればよい. 
もし渦輸送が渦による変位の面において等方的であれば, 三次元の輸送テンソルは, 
%%
\begin{equation}
  \Dvect{S}' = \kappa_s
\begin{pmatrix}
    1 &0 &0 \\
    0 &1 &0 \\
    0 &0 &\alpha 
  \end{pmatrix}
\end{equation}
%%
である. 
運動の傾斜は二次元ベクトル$\Dvect{s} = (s_x,s_y)$である. 
輸送テンソルを物理空間へと回転させれば, 二次元の場合と同様に, 
%%
\begin{eqnarray}
\Dvect{S} &=& \kappa_S 
\begin{pmatrix}
  1+s_y^2 + \alpha s_x^2 &(\alpha-1) s_x s_y   &(1-\alpha)s_x \\
  (\alpha-1) s_x s_y     &1+s_x^2+\alpha s_y^2 &(1-\alpha)s_y \\
  (1-\alpha)s_x          &(1-\alpha)s_y        &a + s^2
\end{pmatrix} 
\\
 &\approx& \kappa_S
\begin{pmatrix}
  1     &0       &s_x \\
  0     &1       &s_y \\
  s_x   &s_y     &a + s^2
\end{pmatrix} 
\end{eqnarray}
を得る. 
ここで, 二段目において$s,\alpha \ll 1$であることを用いて近似を行った. 
また, $s=s_x^2 + s_y^2$である. 

\subsubsection*{渦の変位面}
%%
次に, 輸送係数を発見的に選択する.
そのための二つの基礎を考えよう.

\paragraph{I. 傾圧不安定の線形論の利用}
成長する(Eadyの)傾圧波の最も簡単なモデルにおいて, 平均場の等密度面の傾斜の半分に沿うパーセルの軌跡は,
最大のポテンシャルエネルギーを解放することができる.
よって, $s=s_\rho/2$と仮定する. ここで, $s_\rho$は等密度面の傾斜である.
また, \eqref{eq:EddDiffSymTensor2D}において$\alpha$はゼロとする.
二次元の場合には, このことは,
%%
\begin{equation}
 \Dvect{S} = \kappa_s
 \begin{pmatrix}
  1           &s_\rho/2 \\
  s_\rho/2    &s_\rho^2/4
 \end{pmatrix}
\end{equation}
%%
を与える. よって, 
%%
\begin{subequations}
\begin{equation}
  \overline{v' \phi'} = - \kappa_s \left(\DP{\bar{\phi}}{y} + \dfrac{1}{2}s_\rho \DP{\bar{\phi}}{z} \right), 
\end{equation} 
\begin{equation}
  \overline{w' \phi'} = - \dfrac{1}{2}\kappa_s s_\rho \left(\DP{\bar{\phi}}{y} + \dfrac{1}{2}s_\rho \DP{\bar{\phi}}{z} \right). 
\end{equation} 
\label{eq:tracer_transport_eadyBarocLinearTheory}
\end{subequations}
%%
もし, トレサー$\phi$が温位である(そして単なる受動的トレサーでない)ならば,
そのとき, \eqref{eq:tracer_transport_eadyBarocLinearTheory}は前節の$\kappa_s$の大きさの推定と合わせて,
大気における極向きおよび鉛直上向きの拡散フラックスに対するパラメタリゼーションを構成する. 

\paragraph{II. 中立面に沿った流れ}
%%
流体の内部領域が断熱・定常であるならば, そのとき流体の軌跡は中立面, すなわちポテンシャル密度面あるいは温位面に沿う.
故に, 渦フラックスは平均場の中立面に沿う方向に向くことを仮定し, $s=s_\rho$のように選びたいと思うだろう.
しかし, 断熱的な場合においても, これは必ずしも良い選択とは限らない.
断熱条件における熱力学方程式$d b/dt = 0$(ここで, $b(=-g\delta \rho/\rho_0)$は浮力)から, 浮力の渦成分の分散に対する方程式を導けば,
%%
\begin{equation}
 \begin{split}
  \dfrac{1}{2} \DP{\overline{b'^2}}{t} + \dfrac{1}{2} \bar{\Dvect{u}}\cdot\nabla_z \overline{b'^2}
  + \dfrac{1}{2}\bar{w}\DP{\overline{b'^2}}{z} &+ \overline{\Dvect{u}'b'} \cdot \nabla_z \bar{b}
  + \overline{w'b'}\DP{\bar{b}}{z} \\
 &+ \dfrac{1}{2}\nabla_z \cdot \overline{\Dvect{u}'b'^2} + \dfrac{1}{2}\DP{}{z}\overline{w'b'^2}
  = 0
 \end{split}
 \label{eq:bouyancyVariance}
\end{equation}
%%
を得る. 
東西に一様な基本場と小振幅の波を仮定すれば,
%%
\begin{equation}
    \dfrac{1}{2}\DP{\overline{b'^2}}{t}
  = - \overline{v'b'}\DP{\bar{b}}{y} - \overline{w'b'}\DP{\bar{b}}{z} 
\end{equation}
%%
となる%
\footnote{
東西に一様な基本場を仮定するとき, 準地衡流理論の枠組みにおいて,
($f_0, N^2$の掛かった項を除いて)$\bar{v}, \bar{w}$はゼロと考えて良い. 
}%
.
波が統計的に定常であるならば, そのとき左辺はゼロとなり,
%%
\begin{equation}
  \overline{\Dvect{v}_\chi b'} \cdot \nabla_\chi \bar{b} = 0
\end{equation}
%%
を得る.
ここで, $\chi$は, $x$方向に変動のない子午面上のベクトルであることを示す.
この場合には, 勾配に沿うフラックスは存在しない. 
しかし, 波が成長するならば, そのとき, 
$  \overline{\Dvect{v}_\chi b'} \cdot \nabla_\chi \bar{b} < 0$
であり, もし北半球の場合のように,
$\partial \bar{b}/\partial y < 0$かつ$\overline{v'b'}>0$
であるならば,
%%
\begin{equation}
 \dfrac{\overline{w'b'}}{\overline{v'b'}} >
  - \dfrac{\partial\bar{b}/\partial y}{\partial\bar{b}/\partial z} 
\end{equation}
%%
となる.
よって, たとえ流れが断熱的であっても, 混合の傾斜角は平均場の等密度面の傾斜ほど急でない. 
(この傾斜の結果は, $\overline{v'b'}<0$である場合にも不等号の向きは異なるが成り立つ.)
同様に, もし波が減衰しているのならば, 混合の傾斜角は平均場の等密度面の傾斜より急である.
非均質な流れにおいて, \eqref{eq:bouyancyVariance}の平均流による移流は, 時間依存性に対して同様の役割を果たす. 
より大きな分散の領域への平均流による渦の分散の移流は, その等密度面より小さな傾斜をもつ混合の傾斜角を生じさせるだろう.
より小さな分散の領域へ進入する流れに対しては, その逆となる.
統計的に定常かつ断熱的な線形波の場に対してのみ, 混合の傾斜は等密度面に沿うことが保証される.

それでも, 流体の軌跡が中立面に沿うことを仮定して議論を進めよう.
もし中立面と垂直な拡散がなければ$\alpha=0$であり, 輸送テンソルは,
%%
\begin{equation}
 \Dvect{S} = \kappa_s
 \begin{pmatrix}
  1 & 0 & s_\rho^x \\
  0 & 1 & s_\rho^y \\
  s_\rho^x & s_\rho^y & |s_\rho|^2
 \end{pmatrix}
 ,
\end{equation}
%%
あるいは2次元の場合には,
%%
\begin{equation}
 \Dvect{S} = \kappa_s
 \begin{pmatrix}
  1 & s_\rho \\
  s_\rho & |s_\rho|^2
 \end{pmatrix}
\end{equation}
%%
となる.
この場合には, 渦フラックスは, 
%%
\begin{subequations}
\begin{equation}
  \overline{v' \phi'} = - \kappa_s \left(\DP{\bar{\phi}}{y} + s_\rho \DP{\bar{\phi}}{z} \right), 
\end{equation} 
\begin{equation}
  \overline{w' \phi'} = - \kappa_s s_\rho \left(\DP{\bar{\phi}}{y} + s_\rho \DP{\bar{\phi}}{z} \right)
\end{equation} 
\label{eq:tracer_transport_neutralSurface}
\end{subequations}
%%
と書かれる.
今$\phi$が温位$\theta$であり, 温位面が中立面を定義すると考えよう.
このとき, 温位面に沿った渦の運動は明らかに温位を輸送せず,
\eqref{eq:tracer_transport_neutralSurface}によって定義される拡散は効果を持たない. 
このことは, \eqref{eq:tracer_transport_neutralSurface}に対して,
%%
$$
 s_\rho = - \dfrac{\partial_y \bar{\theta}}{\partial_z \bar{\theta}}
$$
%%
を代入したときに,
%%
\begin{equation}
 \overline{v'\theta'} = 0, \;\;\;\;
 \overline{w'\theta'} = 0
\end{equation}
%%
となることからも確認できる.
現実の海洋では, 塩分の存在により, 温位面とポテンシャル密度面と塩分面は一般的には平行でなく,
中立面に沿った温位や塩分の渦拡散が存在する.
しかし, 二番目のトレサー(塩分)の存在に依存した, 傾圧渦によるフラックスは予期できないために,
中立面に沿った温位や塩分の渦拡散は,
基本場の有効位置エネルギーを解放するような傾圧渦による熱フラックスのパラメタリゼーションを\textbf{与えない}.
そのようなパラメタリゼーションのために, 次に反対称輸送テンソルに目を向ける. 


\subsection{拡散係数テンソルの構造: 反対称輸送テンソル}
反対称輸送テンソルはスキューフラックス(勾配ベクトルの垂直方向のフラックス)あるいは
偽の移流を発生させる. 
二次元(水平-鉛直面)において, 反対称輸送テンソルは, 速やかに, 
%%
\begin{equation}
  \Dvect{A} =
\begin{pmatrix}
  0 &-\kappa'_a \\
  \kappa'_a &0
\end{pmatrix}
\end{equation}
と書ける. 
ここで, $\kappa'_a$は空間・時間的に変化して良く, 流れ自体に依存する. 
また, 輸送の全体的な強さを決定する. 
推測により, 三次元においては, 
%%
\begin{equation}
  \Dvect{A} =
\begin{pmatrix}
  0 &0 &-\kappa_a^{'x} \\
  0 &0 &-\kappa_a^{'y} \\
  \kappa_a^{'x} &\kappa_a^{'y} &0
\end{pmatrix}
\end{equation}
と書ける. 
ここで, 上付き添字は成分を表す. 
今, 都合上ゲージを選択し, $A_{21}=-A_{12}=0$とした. 
残る選択は, 輸送係数の符号と大きさを決定することである. 


\subsection*{断熱的かつポテンシャルエネルギーを減少する渦輸送スキーム}
Gent と McWilliams は, 傾圧渦のフラックスによるトレサーの輸送のための, 
海洋モデルにおけるパラメタリゼーションを示した. 
これは, GM スキームとして知られる. 
以下の二つの性質の保持が, 彼らのスキームの基礎である. 
%%
\begin{itemize}
 \item トレーサのモーメントは保存されなければならない. 特に, 等密度面間の流体の量は保存されなければならない. これは, スキームが密度の勾配を横切って浮力を拡散させないことを示唆する. 
 \item 流れの有効位置エネルギーの量は減少しなければならない. この意味に置いて, このパラメタリゼーションは, 有効位置エネルギーを運動エネルギーに輸送する傾圧不安定の効果を模倣する. 
\end{itemize}
%%
一番目は, 反対称拡散係数テンソルを使うことによって自動的に満たされる. 
二つ目の性質は, 輸送係数を等密度面の傾斜に比例するように選択することによって満たされる. 
その場合には, 
%%
\begin{equation}
  \Dvect{A} = \kappa_a
\begin{pmatrix}
 0 &0 &-s_x \\
 0 &0 &-s_y \\
 s_x &s_y &0
\end{pmatrix}
\label{eq:GM90_AntiSymTensor}
\end{equation}
と書ける. 
ここで, $\Dvect{s}=(s_x, s_y)=\nabla_\rho z=- \nabla_z \rho/(\partial \rho/\partial z)$
は等密度面の傾斜である. 
温度場と塩分場を別々にもつ海洋モデルでは, \eqref{eq:GM90_AntiSymTensor}はそれらそれぞれに適用されるだろう. 
そのとき, 等密度面の傾斜は, 状態方程式を使って決定される. 
この輸送によって示唆される特徴が何かをより簡単に見るために, 塩分のない特別な場合を考えることにしよう. 
このとき, 浮力$b$は熱力学変数のみに依存する. 
等密度面の傾斜は$s=-(b_x/b_z, b_y/b_z)$, 水平渦輸送$\Dvect{F}_h=(F_x, F_y)$は, 
%%
\begin{equation}
 \Dvect{F}_h = -\left(-\kappa_a \Dvect{s} \DP{b}{z} \right)
             = - \kappa \nabla_z b
\end{equation}
%%
によって与えられる. 
正の$\kappa_a$に対して, これは従来の勾配を下る(downgradient)拡散である. 

一方, 鉛直方向の輸送は, 
%%
\begin{equation}
  F_z = - \kappa_a \left(s_x\DP{b}{x} + s_y\DP{b}{y}\right)
      = \kappa_a s^2 \DP{b}{z} 
\end{equation}
%%
によって与えられる. 
ここで, $s^2=\Dvect{s}\cdot\Dvect{s}$. 
このフラックスは勾配を上る向きである(upgradient). 
しかしながら, 合計のスキューフラックスは勾配を下る向きでも上る向きでもない. 

勾配を下る向きの水平フラックスと勾配を上る方向の鉛直フラックスの組み合わせは, 
それぞれの密度間隔内の流体の体積を保存すると同時に, 流れの位置エネルギーを減少させるように
作用する. 
鉛直方向の勾配上向きフラックスは, 有効位置エネルギーを減少させるための必要性の結果である. 
暖かく軽い流体が冷たく重い流体の上に乗っている, 静的安定な状況を考えると, 
勾配下向きの鉛直拡散は流体の重心を上昇させる. 結果, ポテンシャルエネルギーは増加する. 
これは, 傾圧不安定の作用と逆である. 
よって, 鉛直拡散の符号は負でなければならず, 
\eqref{eq:GM90_AntiSymTensor}の構造(よって正の水平拡散係数)と共にこのことは, 
上述した二つの性質の両方を満たすことを可能にする. 
このパラメタリゼーションは, 全エネルギーを保存しない. 
すなわち, ポテンシャルエネルギーは対応する運動エネルギーの増加によってバランスされない. 
むしろ, 散逸によって失われることが仮定される. 
最後に, (スキュー)渦拡散係数の大きさは, 前述の現象論的推定によって決定される. 

\subsection*{渦輸送速度}
%%
よって, 渦輸送速度は, 
\begin{equation}
 \begin{split}
   \Dvect{\tilde{u}} &= - \DP{}{z} (\kappa_a \Dvect{s}), \\
   \tilde{w} &= \nabla_z \cdot (\kappa_a \Dvect{s})
 \end{split}
\end{equation}
%%
と与えられる%
\footnote{
スキュー速度$\Dvect{\tilde{v}}$と反対称拡散係数テンソル$\Dvect{A}$の関係は, 
%%
\begin{equation*}
  \tilde{v}_n = - \nabla_m A_{mn}
\end{equation*}
%%
によって与えられる. 
}. 
また, $\Dvect{A}$と関係する流線関数は, 
%%
\begin{equation}
  \Dvect{\psi} = (-\kappa_a s_y, \kappa_a s_x, 0) = \Dvect{k} \times \kappa_a \Dvect{s}
\end{equation}
%%
によって与えられる%
\footnote{
流線関数と反対称拡散係数テンソル$\Dvect{A}$の関係は, 
$\tilde{v}_n = \epsilon_{lmn}\nabla_l \psi_m$および$\tilde{v}_n = - \nabla_m A_{mn}$より, 
$$
A_{mn} = \epsilon_{mnp} \psi_p
$$
と与えられる. 
}. 

Gent-McWilliams のパラメタリゼーションの実装には, 二つの等価な方法があり, 
一つはスキューフラックスを用い, もう一方は渦輸送速度による移流を使う. 
前者は, 渦フラックスを
\begin{equation}
  \Dvect{F}_{\rm sk} = - \nabla b \times \Dvect{\psi}
\end{equation}
と書く. 
一方, 後者は, 
\begin{equation}
  \Dvect{F}_{\rm ad} = - b \nabla \times \Dvect{\psi}
\end{equation}
と書く. 
発散をとれば, 両者は等価な表現となることに注意されたい. 



\newpage
%%

%% 参考文献
\bibliographystyle{plainnat}
\bibliography{./../../reflist/Dennou-OGCM_reflist}
%%%%%%%%%%%%%%%%%%%%%%%%%%%%%%%%

\end{document}
%%%%%%%%              Text End                  %%%%%%%%
%%%%%%%%%%%%%%%%%%%%%%%%%%%%%%%%%%%%%%%%%%%%%%%%%%%%%%%%
