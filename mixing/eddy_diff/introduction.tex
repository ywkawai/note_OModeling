\section{はじめに}
全球的な海洋循環の計算や気候計算に用いられる海洋大循環モデルでは, 傾圧不安定によって生じる渦(水平スケール O(100 km))は陽に解像することが難しいために, パラメタリゼーションされる. 
一般的に, そのパラメータ化手法として等密度面混合スキーム\citep{redi1982oceanic}とGMスキーム\citep{gent1990isopycnal}が使われることが多い. 
本ノートの目的は, これらのパラメタリゼーションの理論的基礎をまとめ, 等密度面混合スキームや GMスキームの理解を深めることである. 
そのために, \citet{vallis2006atmospheric}に習って, 海洋における傾圧渦による流体の性質の輸送を定式化し, 
その解釈を行う. 

海洋の傾圧渦の輸送の効果は, トレサー(塩分や温位)の時間発展式に拡散項を加えることによって表現される. 
そのため, この拡散項の拡散係数をどのように表現するかが, 傾圧渦の輸送の効果を表現する上で重要である. 
はじめに, 拡散項の一般的な表現を得るために, 拡散係数テンソルを導入する. 
次に, 等密度面混合や傾圧不安定をパラメータ化するときの拡散テンソルの形式を導く. 


