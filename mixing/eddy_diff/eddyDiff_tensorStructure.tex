\subsection{拡散係数テンソルの構造: 対称輸送テンソル}
勾配方向に沿った拡散は, 対称輸送テンソルによる輸送である. 
今, 東西に一様な渦の統計的性質をもつ, 周期的なチャネルにおける輸送を考えよう. 
よって, 平均操作は東西平均である. 
このとき, トレサーの南北輸送, 鉛直輸送の両方に関心がある. 
%%
\begin{equation}
  \overline{v' \phi'} = - \kappa^{vy} \DP{\overline{\phi}}{y}
                        - \kappa^{vz} \DP{\overline{\phi}}{z}, 
\end{equation}
\begin{equation}
  \overline{w' \phi'} = - \kappa^{wy} \DP{\overline{\phi}}{y}
                        - \kappa^{wz} \DP{\overline{\phi}}{z}. 
\end{equation}
%%
ここで, 対称テンソルの仮定により$\kappa^{wy} = \kappa^{vz}$である. 
さまざまな輸送係数の関係は, 等密度面あるいは等エントロピー面の傾斜や
等エントロピー面に対する渦の軌跡の関係の両方によって影響を受けるだろう. 
カーテシアン座標($y-z$座標)において, 輸送テンソルは対角的である必要はないが, 
局所的には拡散係数テンソルが対角的となる自然座標系が必ず存在する. 
そのような座標系において, 拡散係数テンソル$\Dvect{S'}$は, 
%%
\begin{equation}
  \Dvect{S}' = \kappa_s
\begin{pmatrix}
    1 &0  \\
    0 &\alpha
  \end{pmatrix}
\end{equation}
%%
と書ける. 
ここで, $\kappa_S$は全体の大きさを決定する. 
また, $\alpha$は直交する二方向の輸送係数の比である. 
大規模な傾圧渦における流体の変位はほとんど水平であるが, 厳密には水平ではない. 
例えば, 変位は等密度面に沿うか, あるいは水平面と等密度面の間の角度にある. 
拡散係数テンソルが対角的である座標系は, 流体の変位が生じる方向に
沿った面によって定義される座標系であると言っても良い. 
流体の経路に沿う方向の輸送と垂直な方向の輸送はそれぞれ異なる物理現象の結果であるため, 
この主張は賢明である. 
したがって, 輸送テンソルがこの座標系において対角的であることを期待してよい. 

渦による変位は主に水平方向であるので, この座標系の傾斜は水平面に対して小さな角度で傾斜している. 
すなわち, $s = \tan \theta \approx \theta \ll 1$. 
さらに, パラメータ$\alpha$は小さい($\alpha \ll 1$)と思ってよい. 
なぜならば, $\alpha$は渦による流体の運動と垂直な方向の輸送を表現するからである. 
今, $y-z$座標系に変換するために, テンソル$\Dvect{S}$を角度$\theta$だけ回転させる. 
つまり, 
%%
\begin{eqnarray}
  \Dvect{S} &=& \kappa_S
\begin{pmatrix}
  \cos\theta &-\sin\theta \\
  \sin\theta &\cos\theta
\end{pmatrix}
\begin{pmatrix}
  1 &0 \\
  0 &\alpha
\end{pmatrix}
\begin{pmatrix}
  \cos\theta &\sin\theta \\
  -\sin\theta &\cos\theta
\end{pmatrix}
\\
 &\approx& \kappa_S
\begin{pmatrix}
 1+s^2 \alpha &s(1-\alpha) \\
 s(1-\alpha)  &s^2 + \alpha
\end{pmatrix} 
\\
 &\approx& \kappa_S
\begin{pmatrix}
  1 &s \\
  s &s^2+\alpha
\end{pmatrix}
\label{eq:EddDiffSymTensor2D}
\end{eqnarray}%
%
を得る. 
ただし, 二段目において$s \ll 1$であること, 
三段目においてさらに$\alpha \ll 1$であることを用いて近似を行った. 
三次元においても同様の方法をとればよい. 
もし渦輸送が渦による変位の面において等方的であれば, 三次元の輸送テンソルは, 
%%
\begin{equation}
  \Dvect{S}' = \kappa_s
\begin{pmatrix}
    1 &0 &0 \\
    0 &1 &0 \\
    0 &0 &\alpha 
  \end{pmatrix}
\end{equation}
%%
である. 
運動の傾斜は二次元ベクトル$\Dvect{s} = (s_x,s_y)$である. 
輸送テンソルを物理空間へと回転させれば, 二次元の場合と同様に, 
%%
\begin{eqnarray}
\Dvect{S} &=& \kappa_S 
\begin{pmatrix}
  1+s_y^2 + \alpha s_x^2 &(\alpha-1) s_x s_y   &(1-\alpha)s_x \\
  (\alpha-1) s_x s_y     &1+s_x^2+\alpha s_y^2 &(1-\alpha)s_y \\
  (1-\alpha)s_x          &(1-\alpha)s_y        &a + s^2
\end{pmatrix} 
\\
 &\approx& \kappa_S
\begin{pmatrix}
  1     &0       &s_x \\
  0     &1       &s_y \\
  s_x   &s_y     &a + s^2
\end{pmatrix} 
\end{eqnarray}
を得る. 
ここで, 二段目において$s,\alpha \ll 1$であることを用いて近似を行った. 
また, $s=s_x^2 + s_y^2$である. 

\subsubsection*{渦の変位面}
%%
次に, 輸送係数を発見的に選択する.
そのための二つの基礎を考えよう.

\paragraph{I. 傾圧不安定の線形論の利用}
成長する(Eadyの)傾圧波の最も簡単なモデルにおいて, 平均場の等密度面の傾斜の半分に沿うパーセルの軌跡は,
最大のポテンシャルエネルギーを解放することができる.
よって, $s=s_\rho/2$と仮定する. ここで, $s_\rho$は等密度面の傾斜である.
また, \eqref{eq:EddDiffSymTensor2D}において$\alpha$はゼロとする.
二次元の場合には, このことは,
%%
\begin{equation}
 \Dvect{S} = \kappa_s
 \begin{pmatrix}
  1           &s_\rho/2 \\
  s_\rho/2    &s_\rho^2/4
 \end{pmatrix}
\end{equation}
%%
を与える. よって, 
%%
\begin{subequations}
\begin{equation}
  \overline{v' \phi'} = - \kappa_s \left(\DP{\bar{\phi}}{y} + \dfrac{1}{2}s_\rho \DP{\bar{\phi}}{z} \right), 
\end{equation} 
\begin{equation}
  \overline{w' \phi'} = - \dfrac{1}{2}\kappa_s s_\rho \left(\DP{\bar{\phi}}{y} + \dfrac{1}{2}s_\rho \DP{\bar{\phi}}{z} \right). 
\end{equation} 
\label{eq:tracer_transport_eadyBarocLinearTheory}
\end{subequations}
%%
もし, トレサー$\phi$が温位である(そして単なる受動的トレサーでない)ならば,
そのとき, \eqref{eq:tracer_transport_eadyBarocLinearTheory}は前節の$\kappa_s$の大きさの推定と合わせて,
大気における極向きおよび鉛直上向きの拡散フラックスに対するパラメタリゼーションを構成する. 

\paragraph{II. 中立面に沿った流れ}
%%
流体の内部領域が断熱・定常であるならば, そのとき流体の軌跡は中立面, すなわちポテンシャル密度面あるいは温位面に沿う.
故に, 渦フラックスは平均場の中立面に沿う方向に向くことを仮定し, $s=s_\rho$のように選びたいと思うだろう.
しかし, 断熱的な場合においても, これは必ずしも良い選択とは限らない.
断熱条件における熱力学方程式$d b/dt = 0$(ここで, $b(=-g\delta \rho/\rho_0)$は浮力)から, 浮力の渦成分の分散に対する方程式を導けば,
%%
\begin{equation}
 \begin{split}
  \dfrac{1}{2} \DP{\overline{b'^2}}{t} + \dfrac{1}{2} \bar{\Dvect{u}}\cdot\nabla_z \overline{b'^2}
  + \dfrac{1}{2}\bar{w}\DP{\overline{b'^2}}{z} &+ \overline{\Dvect{u}'b'} \cdot \nabla_z \bar{b}
  + \overline{w'b'}\DP{\bar{b}}{z} \\
 &+ \dfrac{1}{2}\nabla_z \cdot \overline{\Dvect{u}'b'^2} + \dfrac{1}{2}\DP{}{z}\overline{w'b'^2}
  = 0
 \end{split}
 \label{eq:bouyancyVariance}
\end{equation}
%%
を得る. 
東西に一様な基本場と小振幅の波を仮定すれば,
%%
\begin{equation}
    \dfrac{1}{2}\DP{\overline{b'^2}}{t}
  = - \overline{v'b'}\DP{\bar{b}}{y} - \overline{w'b'}\DP{\bar{b}}{z} 
\end{equation}
%%
となる%
\footnote{
東西に一様な基本場を仮定するとき, 準地衡流理論の枠組みにおいて,
($f_0, N^2$の掛かった項を除いて)$\bar{v}, \bar{w}$はゼロと考えて良い. 
}%
.
波が統計的に定常であるならば, そのとき左辺はゼロとなり,
%%
\begin{equation}
  \overline{\Dvect{v}_\chi b'} \cdot \nabla_\chi \bar{b} = 0
\end{equation}
%%
を得る.
ここで, $\chi$は, $x$方向に変動のない子午面上のベクトルであることを示す.
この場合には, 勾配に沿うフラックスは存在しない. 
しかし, 波が成長するならば, そのとき, 
$  \overline{\Dvect{v}_\chi b'} \cdot \nabla_\chi \bar{b} < 0$
であり, もし北半球の場合のように,
$\partial \bar{b}/\partial y < 0$かつ$\overline{v'b'}>0$
であるならば,
%%
\begin{equation}
 \dfrac{\overline{w'b'}}{\overline{v'b'}} >
  - \dfrac{\partial\bar{b}/\partial y}{\partial\bar{b}/\partial z} 
\end{equation}
%%
となる.
よって, たとえ流れが断熱的であっても, 混合の傾斜角は平均場の等密度面の傾斜ほど急でない. 
(この傾斜の結果は, $\overline{v'b'}<0$である場合にも不等号の向きは異なるが成り立つ.)
同様に, もし波が減衰しているのならば, 混合の傾斜角は平均場の等密度面の傾斜より急である.
非均質な流れにおいて, \eqref{eq:bouyancyVariance}の平均流による移流は, 時間依存性に対して同様の役割を果たす. 
より大きな分散の領域への平均流による渦の分散の移流は, その等密度面より小さな傾斜をもつ混合の傾斜角を生じさせるだろう.
より小さな分散の領域へ進入する流れに対しては, その逆となる.
統計的に定常かつ断熱的な線形波の場に対してのみ, 混合の傾斜は等密度面に沿うことが保証される.

それでも, 流体の軌跡が中立面に沿うことを仮定して議論を進めよう.
もし中立面と垂直な拡散がなければ$\alpha=0$であり, 輸送テンソルは,
%%
\begin{equation}
 \Dvect{S} = \kappa_s
 \begin{pmatrix}
  1 & 0 & s_\rho^x \\
  0 & 1 & s_\rho^y \\
  s_\rho^x & s_\rho^y & |s_\rho|^2
 \end{pmatrix}
 ,
\end{equation}
%%
あるいは2次元の場合には,
%%
\begin{equation}
 \Dvect{S} = \kappa_s
 \begin{pmatrix}
  1 & s_\rho \\
  s_\rho & |s_\rho|^2
 \end{pmatrix}
\end{equation}
%%
となる.
この場合には, 渦フラックスは, 
%%
\begin{subequations}
\begin{equation}
  \overline{v' \phi'} = - \kappa_s \left(\DP{\bar{\phi}}{y} + s_\rho \DP{\bar{\phi}}{z} \right), 
\end{equation} 
\begin{equation}
  \overline{w' \phi'} = - \kappa_s s_\rho \left(\DP{\bar{\phi}}{y} + s_\rho \DP{\bar{\phi}}{z} \right)
\end{equation} 
\label{eq:tracer_transport_neutralSurface}
\end{subequations}
%%
と書かれる.
今$\phi$が温位$\theta$であり, 温位面が中立面を定義すると考えよう.
このとき, 温位面に沿った渦の運動は明らかに温位を輸送せず,
\eqref{eq:tracer_transport_neutralSurface}によって定義される拡散は効果を持たない. 
このことは, \eqref{eq:tracer_transport_neutralSurface}に対して,
%%
$$
 s_\rho = - \dfrac{\partial_y \bar{\theta}}{\partial_z \bar{\theta}}
$$
%%
を代入したときに,
%%
\begin{equation}
 \overline{v'\theta'} = 0, \;\;\;\;
 \overline{w'\theta'} = 0
\end{equation}
%%
となることからも確認できる.
現実の海洋では, 塩分の存在により, 温位面とポテンシャル密度面と塩分面は一般的には平行でなく,
中立面に沿った温位や塩分の渦拡散が存在する.
しかし, 二番目のトレサー(塩分)の存在に依存した, 傾圧渦によるフラックスは予期できないために,
中立面に沿った温位や塩分の渦拡散は,
基本場の有効位置エネルギーを解放するような傾圧渦による熱フラックスのパラメタリゼーションを\textbf{与えない}.
そのようなパラメタリゼーションのために, 次に反対称輸送テンソルに目を向ける. 


\subsection{拡散係数テンソルの構造: 反対称輸送テンソル}
反対称輸送テンソルはスキューフラックス(勾配ベクトルの垂直方向のフラックス)あるいは
偽の移流を発生させる. 
二次元(水平-鉛直面)において, 反対称輸送テンソルは, 速やかに, 
%%
\begin{equation}
  \Dvect{A} =
\begin{pmatrix}
  0 &-\kappa'_a \\
  \kappa'_a &0
\end{pmatrix}
\end{equation}
と書ける. 
ここで, $\kappa'_a$は空間・時間的に変化して良く, 流れ自体に依存する. 
また, 輸送の全体的な強さを決定する. 
推測により, 三次元においては, 
%%
\begin{equation}
  \Dvect{A} =
\begin{pmatrix}
  0 &0 &-\kappa_a^{'x} \\
  0 &0 &-\kappa_a^{'y} \\
  \kappa_a^{'x} &\kappa_a^{'y} &0
\end{pmatrix}
\end{equation}
と書ける. 
ここで, 上付き添字は成分を表す. 
今, 都合上ゲージを選択し, $A_{21}=-A_{12}=0$とした. 
残る選択は, 輸送係数の符号と大きさを決定することである. 

