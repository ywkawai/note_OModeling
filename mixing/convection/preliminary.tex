\section{外洋における対流の様相}%
本節では, 
外洋における対流過程についての観測事実, 理論, モデルについてのレビュー(\cite{marshall1999open})を参考に, 
海洋の対流の様相について簡単にまとめる. 

対流の過程は間欠的であり, 階層的なスケールを持つことが,
観測的事実から示唆されている.
対流の過程は三段階に区別することができ,
図\ref{fig:ooconvection_phase}のように概略される. 
%%
\begin{figure}
 \includegraphics{ooconvection.png}
 \caption{外洋の対流(open ocean convection)の模式図. Marshall and Schott(1999)の図3を引用.(a)準備的段階, (b)深い対流, (c)水平混合と分散.
 巻き矢印は海面の浮力フラックスを表す.
 また, 成層や等密度面の露出(outcrop)を実線によって.
 対流によって混合した流体塊を陰影で表している.}
 \label{fig:ooconvection_phase}
\end{figure}
%%
準備的段階(図\ref{fig:ooconvection_phase}a)の間,
「ドーム」状の等密度面を伴う環流スケールの低気圧性の循環は,
成層の弱い内部域の水を海面近くに持ってくる.
そのとき, 卓越する気象条件と関係した浮力強制は, 対流を引き起こす.
冬季に入ると, 激しい浮力の損失が, 低気圧性のドームの海面近傍の成層を侵食する.
その後の冷却は, 深い対流を開始する.
そこでは, 数多くのプリュームにおいて流体カラムの大部分は転覆し(図\ref{fig:ooconvection_phase}b),
海面の密度の大きい海水は, 鉛直方向に配送される. 
プリュームの水平スケールのオーダーは, $\le 1$km であり,
その鉛直速度は, 10cm/s に達する. 
プリュームは, 準備的段階で中立的になった領域に渡って,
海洋の性質(例えば, 塩分や温位)を急速に混合し,
半径数10kmから100km超のスケールを伴う深い「混合されたパッチ」を形成すると考えられている.
このパッチは, チムニー(chimney)と呼ばれる.

混合されたパッチは水平方向に流体をその周辺の流体と交換するので. 
強い強制の停止, あるいは冷却が何日にも渡って継続するならば, 
対流スケールの卓越的な鉛直熱輸送は, 地衡流渦と関係した水平輸送に取って代わられる(図\ref{fig:ooconvection_phase}c).
個々の渦は, 対流してきた海水を, 地衡流バランスにあるコーヒレントなレンズへと組織化する傾向がある.
混合した流体は重力や回転の効果の下で分散し, 等ポテンシャル密度面に沿って広がる.
そして, 数週間から数ヶ月の時間スケールで, パッチの衰弱と周囲の成層した流体による再占有に至る. 