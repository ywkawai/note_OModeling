\section{非貫入型対流調節}
%%
対流プリュームは, 温度・塩分・密度といった特性を効率的に混合する.
「対流調節」は, 大循環モデルにおける静的不安定を取り除くためによく用いられる.
対流調節にはいくつかの数値的手法が存在するが, 全ての対流調節のアルゴリズムは同じ原理に基づいている.
すなわち, あるモデルの鉛直レベルにおけるポテンシャル密度と
すぐ下の鉛直レベルにおけるポテンシャル密度を比較し, もし上側の方が
ポテンシャル密度が大きければ, パーセルの鉛直混合が発生する.

最初の対流調節の実装(Box,1969; Cox, 1984)は, 鉛直密度一様な状態を得るために,
繰り返しの過程の中ですぐ隣り合う鉛直レベルだけを比較する.
しかし, より最近では, 海水カラムの静的不安定な部分の全体を混合する(Marotzke,1991).
これらの対流調節の実装において, 海水カラムは(1 時間スッテプの内に)瞬時に調節される.
一方で, 静的不安定な箇所で強い鉛直拡散係数を与え, 対流の過程を鉛直拡散によって表現する手法
(enhanced vertical diffusion)もまた広く使われている(Cox, 1984; Marotzke, 1991).
鉛直拡散による対流過程のパラメタリゼーションは, 調節の時間スケールが瞬間的でなく有限である
「対流調節」と, 形式的には同じである(Klinger et al.,1996).

\subsection{「瞬間的」な対流調節}
%%
簡潔には, 海水カラム中に静的不安定な部分が存在する場合に,
ポテンシャル密度の鉛直構造が中立安定になるまで, 静的不安定な部分を下方に向かって瞬時に混合する.
この対流調節のアルゴリズムは, 以下のような手法を用いる繰り返しの過程である.
%%
\begin{enumerate}
 \item 全格子点おけるポテンシャル密度を計算する.
 \item 静的不安定な箇所を探すために,隣接する鉛直レベルのペア全てに対して,ポテンシャル密度を比較する.
 \item 最も海面に近い静的不安定な鉛直レベルのペアを混合する.
 \item 一つ下の鉛直格子点のポテンシャル密度を確認する. 3. で計算された混合後のポテンシャル密度よりも小さければ, 三層全て混合する. 静的安定な鉛直レベルに達するまで, この手順を続ける.
 \item 新しく混合した部分のすぐ上の鉛直レベルに戻り, 今そのペアが静的不安定になったかどうかを確認する. もしそうならば, その鉛直レベルを混合対象のレベルに含め 3. に戻る. そうでなければ,たった今混合行った部分より下側に存在する, 静的不安定なペアを検索し, 発見したら 3. に戻る. 検索が海底に達したら, アルゴリズムを終了する. 
\end{enumerate}
%%
図\ref{fig:convective_adjust_instant}を用いて, このアルゴリズムを直感的に説明する.
今, 初期のポテンシャル密度の分布が, 黒色の実線で与えられると考えよう. 
海面から静的不安定な鉛直レベルのペアを検索すると, 最初の静的不安定な鉛直レベルのペアは k と k+1の間で発見される.
kとk+1の2つのレベルの温位と塩分は, カラムの熱と塩分が保存するように鉛直方向に混合する.
もし新しく計算されたポテンシャル密度においてk+1とk+2の間で静的不安定であるならば,
k,k+1,k+2の間で混合を行う.
このプロセスを, レベル k+2 より下の静的安定が確証されるまで繰り返す(iteration1).
その後, k-1とkの間のポテンシャル密度を比較し静的不安定か否かを調べ, もしそうならば iteration1 と同様の手順を行う(iteration2).
図において, iteration1, iteration2 後のポテンシャル密度分布はそれぞれ,
緑色と赤色の実線で示される.

%%
\begin{figure}
  \begin{center}
   \includegraphics[width=12cm,height=10cm]{convective_adjust_fig.png}
   \label{fig:convective_adjust_instant}
   \caption{(瞬間的な)対流調節のアルゴリズムにおける静的不安定の取り扱い.}
   \end{center}
\end{figure}
%%

\subsection{enhanced vertical diffusion と「遅い」対流調節}
enhanced vertical diffusion や Klinger et al.(1996)の「遅い」対流調節では,
海洋の対流の時間スケールを考慮する.
したがって, 「瞬間的な」対流調節とは異なり, 一時間ステップ中に海水コラムの静的不安定を完全に解消されるとは限らない.

対流による混合は, 時間スケール$t_{\rm mix}$によって特徴付けられる.
ここで, $t_{\rm mix}$は混合層の底の海水の特性が, 混合層上端の浮力強制によって大きく変化するまでにかかる時間スケールである.
流体粒子は対流プリューム中で海面から混合層の底まで下降するので,
この時間スケールは粒子の輸送時間と関係づけられる.
鉛直速度$w_{\rm plume}$によって特徴付けられる対流プリュームに対して,
$t_{\rm mix}$のオーダーは, 
%%
\begin{equation}
  t_{\rm mix} \approx h/w_{\rm plume}
\end{equation}
%%
と取られる.
深い対流の典型的な鉛直速度( 3$\sim$10cm/s)に対して, $t_{\rm mix}$は半日かそれより長くなるため,
$t_{\rm mix}$がゼロである「瞬間的な」対流調節には, 再考の余地がある. 

Klinger et al.(1996)は,有限の$t_{\rm mix}$を表現することを試みた,簡単な対流のパラメタリゼーションを議論した.
静的不安定な領域における温度や塩分の鉛直拡散係数の大きさは, $t_{\rm mix}$の物理的な知識から推定することが望ましい.
拡散系において, シグナルを距離$h$だけ送るには$h^2/\kappa$のオーダーの時間がかかるので,
拡散係数の大きさは, 
 %%
 \begin{equation}
  \kappa \approx h^2/t_{\rm mix} = h w_{\rm plume}
   \label{eq:vertical_diffusivity}
 \end{equation}
 %%
 と見積もられる.
 例えば, $w_{\rm plume}=0.05$cm/s, $h=1000$mととれば, $\kappa=50$m$^2$/s となる.

 「遅い」対流調節のアルゴリズムは, 以下のようなものである.
 ただし, ここでは簡単化のために, 鉛直レベルの間隔は等間隔であるとする.
 %%
 \begin{enumerate}
  \item 全格子点におけるポテンシャル密度を計算する.
  \item 隣接する鉛直レベルのペア全てに対しポテンシャル密度を比較し, 最も深部にある静的不安定なペアを検索する. 
  \item 2. で発見したペアより下の$N_{\rm mix}$層に渡って混合を行う. ここで, $N_{\rm mix}$は\eqref{eq:vertical_diffusivity}の鉛直拡散係数から決定される. 今, 鉛直レベルのペアのインデックスを$k,k+1$とし, それぞれのポテンシャル密度を$\rho_{K}=\rho_{K+1} + \Delta \rho$, $\rho_{K+1}$と書くとする. また, 鉛直レベル$k+1$の上にある同じポテンシャル密度を持つ層数を$N_H$とする. このとき, 混合後のポテンシャル密度は次のように計算する.
  \begin{equation}
   \rho_{k,{\rm new}} = \rho_{k,{\rm old}} - \Delta \rho + \dfrac{N_H}{N_H + N_{\rm mix}}\Delta \rho, \;\;\;\;\;
    ({\rm for} \;\; k=K-N_H+1, \dots, K)
  \end{equation}
  \begin{equation}
   \rho_{k,{\rm new}} = \rho_{k,{\rm old}} + \dfrac{N_H}{N_H + N_{\rm mix}}\Delta \rho \;\;\;\;\;
    ({\rm for} \;\; k=K+1, \dots, K+N_{\rm mix})
  \end{equation}
 \item 鉛直レベル$K-N_H+1$より上に存在する次の静的不安定なペアを対象にして, 3. の手順に戻る. もし海面に最も近い鉛直レベルに達したら, アルゴリズムを終了する.
 \end{enumerate}
 %%
3. のポテンシャル密度の再分配は, 図\ref{fig:SlowConvAdjustFig}において直感的に理解できる. 

%%
 \begin{figure}
  \begin{center}
   \includegraphics[width=15cm]{slow_convective_adjust_fig.png}
   \caption{「遅い」対流調節における密度の再分配の模式図. }
   \label{fig:SlowConvAdjustFig}
  \end{center}
 \end{figure}
 %%